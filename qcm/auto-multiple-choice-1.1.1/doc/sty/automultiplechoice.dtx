% \iffalse meta-comment
%
% Copyright (C) 2008-2012 Alexis Bienvenue
%
% This file is part of Auto-Multiple-Choice
%
% Auto-Multiple-Choice is free software: you can redistribute it
% and/or modify it under the terms of the GNU General Public License
% as published by the Free Software Foundation, either version 2 of
% the License, or (at your option) any later version.
%
% Auto-Multiple-Choice is distributed in the hope that it will be
% useful, but WITHOUT ANY WARRANTY; without even the implied warranty
% of MERCHANTABILITY or FITNESS FOR A PARTICULAR PURPOSE.  See the GNU
% General Public License for more details.
%
% You should have received a copy of the GNU General Public License
% along with Auto-Multiple-Choice.  If not, see
% <http://www.gnu.org/licenses/>.
%
% $Revision: 431 $
%
% \fi
% \iffalse
%<package>\NeedsTeXFormat{LaTeX2e}
%<package>\ProvidesPackage{automultiplechoice}
%
%<*batchfile>
\begingroup
\input docstrip.tex
\keepsilent
\preamble

Copyright (C) 2008-2012 Alexis Bienvenue

This file is part of Auto-Multiple-Choice

Auto-Multiple-Choice is free software: you can redistribute it
and/or modify it under the terms of the GNU General Public License
as published by the Free Software Foundation, either version 2 of
the License, or (at your option) any later version.

Auto-Multiple-Choice is distributed in the hope that it will be
useful, but WITHOUT ANY WARRANTY; without even the implied warranty
of MERCHANTABILITY or FITNESS FOR A PARTICULAR PURPOSE.  See the GNU
General Public License for more details.

You should have received a copy of the GNU General Public License
along with Auto-Multiple-Choice.  If not, see
<http://www.gnu.org/licenses/>.

\endpreamble
\askforoverwritefalse
\generate{\file{automultiplechoice.sty}{\from{automultiplechoice.dtx}{package}}}
\endgroup
%</batchfile>
%
%<*driver>
\documentclass{ltxdoc}
\usepackage[utf8]{inputenc}
\usepackage[nopage,calibration]{automultiplechoice}
\usepackage{verbatim}
\usepackage{fp}
\usepackage{multicol}
\usepackage[linenumberformat={{}}]{examplep}
\usepackage{codep}
\usepackage[hyperindex=false]{hyperref}
\usepackage{pdfpages}
\EnableCrossrefs
\CodelineIndex
\RecordChanges
\begin{document}
% UTF8 unkonwn characters are replaced by diamonds
\makeatletter\def\UTFviii@defined#1{%
  \ifx#1\relax%
      \(\diamond\)%
  \else\expandafter
    #1%
  \fi
}\makeatother
\DocInput{automultiplechoice.dtx}
\end{document}
%</driver>
% \fi
%
% \CheckSum{0}
%
% \CharacterTable
% {Upper-case      \A\B\C\D\E\F\G\H\I\J\K\L\M\N\O\P\Q\R\S\T\U\V\W\X\Y\Z
%  Lower-case      \a\b\c\d\e\f\g\h\i\j\k\l\m\n\o\p\q\r\s\t\u\v\w\x\y\z
%  Digits          \0\1\2\3\4\5\6\7\8\9
%  Exclamation     \!     Double quote  \"     Hash (number) \#
%  Dollar          \$     Percent       \%     Ampersand     \&
%  Acute accent    \'     Left paren    \(     Right paren   \)
%  Asterisk        \*     Plus          \+     Comma         \,
%  Minus           \-     Point         \.     Solidus       \/
%  Colon           \:     Semicolon     \;     Less than     \<
%  Equals          \=     Greater than  \>     Question mark \?
%  Commercial at   \@     Left bracket  \[     Backslash     \\
%  Right bracket   \]     Circumflex    \^     Underscore    \_
%  Grave accent    \`     Left brace    \{     Vertical bar  \|
%  Right brace     \}     Tilde         \~}
%
%
% \changes{v0.376}{2011/01/11}{Old color using mechanism}
% \changes{v0.426}{2011/02/08}{Using now savepos for layout detection}
%
% \DoNotIndex{\',\\,\`,\,,\@empty,\@expandtwoargs}
% \DoNotIndex{\bgroup,\egroup,\clearpage,\closeout,\PackageInfo,\parindent}
% \DoNotIndex{\@ifpackageloaded,\z@,\@ne,\tw@,\m@ne,\@undefined}
% \DoNotIndex{\@unexpandable@protect,\@whilenum,\addtocounter,\advance,\Alph,\@Alph}
% \DoNotIndex{\arabic,\AtBeginDocument,\IfFileExists}
% \DoNotIndex{\AtBeginPage,\begin,\end,\begingroup,\endgroup,\textbf,\BODY}
% \DoNotIndex{\box,\circle,\clubsuit,\csname,\DeclareFontShape}
% \DoNotIndex{\DeclareOptionX,\def,\define@key,\divide,\do,\dp,\edef,\else}
% \DoNotIndex{\emph,\endcsname,\eTeXversion,\expandafter,\fancyfoot}
% \DoNotIndex{\fancyhead,\fancyhf,\fancyhfoffset,\fboxrule,\fboxsep}
% \DoNotIndex{\fcolorbox,\fi,\fill,\fontencoding,\fontfamily,\fontseries}
% \DoNotIndex{\fontshape,\fontsize,\FPeval,\FPiflt,\FPset,\geometry,\global}
% \DoNotIndex{\hbox,\headrulewidth,\hfill,\hspace,\hss,\ifFileExists,\ifnum}
% \DoNotIndex{\ifodd,\ifx,\ignorespaces,\immediate,\index,\item,\itemsep}
% \DoNotIndex{\jobname,\label,\let,\leavevmode,\linewidth,\loop,\lower,\LR,\makebox,\mbox}
% \DoNotIndex{\message,\multiply,\newcommand,\newcount,\newcounter,\newdimen}
% \DoNotIndex{\NewEnviron,\newenvironment,\newif,\newlength,\newtoks}
% \DoNotIndex{\newwrite,\noexpand,\noindent,\number,\openout,\PackageWarning}
% \DoNotIndex{\page,\pagestyle,\paperheight,\paperwidth,\par,\pdflastxpos}
% \DoNotIndex{\pdflastypos,\pdfsavepos,\ProcessOptionsX,\protect}
% \DoNotIndex{\protected@write,\providecommand,\put,\ref,\relax}
% \DoNotIndex{\renewcommand,\repeat,\RequirePackage,\romannumeral,\rotatebox}
% \DoNotIndex{\sc,\selectfont,\setbox,\setcounter,\setkeys,\setlength}
% \DoNotIndex{\string,\strip@pt,\subsection,\textcolor,\the,\thepage}
% \DoNotIndex{\thinlines,\time,\texttt,\undefined,\unitlength,\usepackage,\vbox}
% \DoNotIndex{\vfill,\vspace,\vss,\write}
% \DoNotIndex{\AMC@postcorrecttrue}\DoNotIndex{\AMC@postcorrectfalse}
%
% \DoNotIndex{\AMC@watermarktrue}
% \DoNotIndex{\AMC@pagelayouttrue}
% \DoNotIndex{\AMCformulaire@dedanstrue}
% \DoNotIndex{\AMC@zoneformulairetrue}
% \DoNotIndex{\AMCformulaire@dedanstrue}
% \DoNotIndex{\AMC@rbloctrue}
% \DoNotIndex{\AMCune@bonnetrue}
% \DoNotIndex{\AMC@qbloctrue}
% \DoNotIndex{\AMCtype@multitrue}
% \DoNotIndex{\AMC@ordretrue}
% \DoNotIndex{\AMC@correcheadtrue}
% \DoNotIndex{\AMC@correctrue}
% \DoNotIndex{\AMC@correctrue}
% \DoNotIndex{\AMC@qbloctrue}
% \DoNotIndex{\AMC@ensembletrue}
% \DoNotIndex{\AMC@inside@boxtrue}
% \DoNotIndex{\AMC@inside@boxfalse}
% \DoNotIndex{\AMC@outside@boxtrue}
% \DoNotIndex{\AMC@outside@boxfalse}
% \DoNotIndex{\AMC@inside@digittrue}
% \DoNotIndex{\AMC@ordretrue}
% \DoNotIndex{\AMC@correcheadtrue}
% \DoNotIndex{\AMC@correctrue}
% \DoNotIndex{\AMC@correcheadtrue}
% \DoNotIndex{\AMC@ordretrue}
% \DoNotIndex{\AMC@correctrue}
% \DoNotIndex{\AMC@qbloctrue}
% \DoNotIndex{\AMCcomplete@multitrue}
% \DoNotIndex{\AMC@ensembletrue}
% \DoNotIndex{\AMC@inside@digittrue}
% \DoNotIndex{\AMC@calibrationtrue}
% \DoNotIndex{\AMC@correcheadtrue}
% \DoNotIndex{\AMC@correctrue}
% \DoNotIndex{\AMC@ordretrue}
% \DoNotIndex{\AMC@affichekeystrue}
% \DoNotIndex{\AMC@plaintrue}
% \DoNotIndex{\AMC@calibrationtrue}
% \DoNotIndex{\AMC@calibrationtrue}
% \DoNotIndex{\AMC@correcheadtrue}
% \DoNotIndex{\AMC@correctrue}
% \DoNotIndex{\AMC@correctrue}
% \DoNotIndex{\AMC@ordrefalse}
% \DoNotIndex{\AMC@correcheadfalse}
% \DoNotIndex{\AMC@affichekeysfalse}
% \DoNotIndex{\AMC@correcfalse}
% \DoNotIndex{\AMC@qblocfalse}
% \DoNotIndex{\AMC@rblocfalse}
% \DoNotIndex{\AMCcomplete@multifalse}
% \DoNotIndex{\AMC@calibrationfalse}
% \DoNotIndex{\AMC@plainfalse}
% \DoNotIndex{\AMC@ensemblefalse}
% \DoNotIndex{\AMC@inside@digitfalse}
% \DoNotIndex{\AMCformulaire@dedansfalse}
% \DoNotIndex{\AMC@qblocfalse}
% \DoNotIndex{\AMCtype@multifalse}
% \DoNotIndex{\AMCformulaire@dedansfalse}
% \DoNotIndex{\AMCune@bonnefalse}
% \DoNotIndex{\AMCtype@multifalse}
% \DoNotIndex{\AMC@zoneformulairefalse}
% \DoNotIndex{\AMC@correcfalse}
% \DoNotIndex{\AMC@watermarkfalse}
% \DoNotIndex{\AMC@watermarkfalse}
% \DoNotIndex{\AMC@pagelayoutfalse}
% \DoNotIndex{\AMC@correcfalse}
% \DoNotIndex{\AMC@correcheadfalse}
% \DoNotIndex{\AMC@watermarkfalse}
% \DoNotIndex{\AMC@correcfalse}
% \DoNotIndex{\AMC@correcheadfalse}
% \DoNotIndex{\AMC@watermarkfalse}
% \DoNotIndex{\AMC@calibrationfalse}
% \DoNotIndex{\AMC@watermarkfalse}
% \DoNotIndex{\AMC@watermarkfalse}
% \DoNotIndex{\AMC@watermarkfalse}
%
% \makeatletter
% \def\SpecialOptIndex#1{\@bsphack
%    \index{#1\actualchar{\protect\ttfamily#1}
%           (option)\encapchar usage}%
%    \index{options:\levelchar#1\actualchar{\protect\ttfamily#1}\encapchar
%           usage}\@esphack}
% \makeatother
% \def\amccode{\vspace{.8ex}\par}
% \begingroup\catcode`\|=11\gdef\Vbar{|}\endgroup
%
% \title{The \textsf{automultiplechoice} package\thanks{This document
% corresponds to version $Revision: 431 $ from AMC 1.1.1}}
% \author{Alexis Bienven\"ue \\ \texttt{paamc@passoire.fr}}
%
% \maketitle
%
% \begin{abstract}
%   This package helps designing multiple choice exams ready for automated
%   marking from papers scans. 
%
%   Answers and questions are optionaly shuffled, creating different sheets
%   for every student.
% \end{abstract}
%
% \section{Introduction}
%
% The package \textsf{automultiplechoice} helps formatting multiple
% choice questionnaries with automated marking from papers scans in
% mind:
% \begin{itemize}
% \item The package can produce different copies of the question sheet
%   for each student, optionaly shuffling answers and questions for
%   each student.
% \item Markers can be printed on each sheet, so as to be able to
%   analyse scans after examination. All the needed information about
%   the position of the markers and the boxes to be checked by the
%   students is given in an auxiliary file during \LaTeX{} run.
% \end{itemize}
%
% See Auto Multiple Choice (AMC) software
% (\url{http://home.gna.org/auto-qcm/}) for an integration of this
% package, with user interface for automated marking.
%
%
% \section{Samples}\label{d:samples}
%
% We begin with several samples to see what can be done with the
% \textsf{automultiplechoice} package. All \textsf{automultiplechoice} commands
% and options will be detailed further.
%
% For all these samples, two sets of questions are used: a group of
% geography questions, and a group of history questions. These are
% defined in a common \LaTeX{} file named |questions.tex|:
% \verbatiminput{questions.tex}
% 
% We will ask \textsf{automultiplechoice} package to include two
% geography questions and two history questions at random for each
% student, shuffling questions and answers, with the following code:
%\iffalse
%<*doc>
%\fi
\begin{verbatim}
\cleargroup{all}
\shufflegroup{geography}
\copygroup[2]{geography}{all}
\shufflegroup{history}
\copygroup[2]{history}{all}
\shufflegroup{all}
\insertgroup{all}
\end{verbatim}
%\iffalse
%</doc>
%\fi
% You can read these commands as ``clear group |all|, shuffle
% questions inside group |geography| and copy the first two to group
% |all|, do the same for group |history|, shuffle the four questions
% copied into |all| and print them''.
%
%
% \subsection{Standard layout}\label{s:amc}
%
% A set of 30 students sheets can be produced from the following
% \LaTeX{} source named \texttt{sample-amc.tex}:
% \verbatiminput{sample-amc.tex} producing a 30-pages document (every
% page has number 1), from which we show the first pages on
% page~\pageref{p:amc}.
%
% Note that ``DRAFT'' indications can be cancelled using option {\tt
% nowatermark}\SpecialOptIndex{nowatermark}, or using AMC software.
% 
% You can see on each page markers that can be used for automated
% completed answer sheets scans analysis:
% \begin{itemize}
% \item Four circles \makeatletter\m@rqueCalage\makeatother{} are
%   printed in the corners, to be able to analyse any rotation or
%   scaling of the scans.
% \item Binary boxes are printed in the header area, so as to be able
%   to read student sheet number and page number. On page~2 for
%   example, you can see that these binary boxes are coding |2/1/59|:
%   \makeatletter
%   \begin{center}
%     \begin{minipage}[b]{\AMC@CBtaille}
%       \AMCbin@begin{1}\noindent%
%       \AMC@binaryBoxes[\AMC@NCBetud]{2}\\
%       \AMCbin@begin{2}\noindent%
%       \AMC@binaryBoxes[\AMC@NCBpage]{1}\ignorespaces%
%       \AMCbin@begin{3}\AMC@binaryBoxes[\AMC@NCBcheck]{59}%
%     \end{minipage}
%     \hbox to 4cm{\hspace*{\fill}%
%     \texttt{+2/1/59+}}
%   \end{center}
%   \makeatother
%   Here, |2| is the student sheet number, |1| is the page
%   number for this student, and |59| is a checking value that can be
%   used for checking correct identification from a scan.
% \end{itemize}
%
% If you also use |calibration| option\SpecialOptIndex{calibration},
% \textsf{automultiplechoice} will produce a |.xy| file with informations
% about the exact position in the page of all the markers, and all the
% boxes. This option is automatically set by AMC software, which then
% use the information in the |.xy| file for automated marking.
%
% \subsection{Separate answer sheet}\label{s:separate}
%
% In some situations, you may need a separate answer sheet:
% \begin{itemize}
% \item this makes cheating even more dificult;
% \item this can reduce the number of pages to scan.
% \end{itemize}
%
% This is done using |separateanswersheet|
% option\SpecialOptIndex{separateanswersheet} of
% \textsf{automultiplechoice} package. You also have to use commands
% |\AMCformBegin| to indicate the beginning of this separate answer
% sheet (usually after a |\clearpage| or |\AMCcleardoublepage|
% command), and |\AMCform| to insert the form to be completed by the
% students, as in the following example (|sample-separate.tex|):
% \verbatiminput{sample-separate.tex}
%
% First pages of the result are shown on
% page~\pageref{p:separate}. There are now 2 pages per student: the
% first with questions, and the second for answers. Only the second
% will be completed by the students, and scanned for analysis.
%
% \subsection{Without markers}\label{s:plain}
%
% With the |nopage| option\SpecialOptIndex{nopage}, package
% \textsf{automultiplechoice} does not include any page markers for scan
% processing. I'm afraid you can't use any automated marking software
% with this layout, but you can still use answer sheet and corrected
% answer sheet (option |indivanswers|\SpecialOptIndex{indivanswers},
% added here) for a manual marking...
%
% The \LaTeX{} source \texttt{sample-plain.tex} that only differs from
% \texttt{sample-amc.tex} by its options passed to
% \textsf{automultiplechoice}:
%\iffalse
%<*doc>
%\fi
\begin{verbatim}
\usepackage[nopage,indivanswers]{automultiplechoice}
\end{verbatim}
%\iffalse
%</doc>
%\fi
% produces a 30-pages document, from which we show the first pages on
% page~\pageref{p:plain}.
%
% \newcommand{\isample}[1]{\IfFileExists{sample-#1.pdf}{\includepdf[pages={1-4},nup=2x2,frame=true,delta=.3cm .3cm,pagecommand={\textbf{First pages from \LaTeX{} source detailed in section~\ref{s:#1} -- see \texttt{sample-#1.pdf}}\label{p:#1}},scale=.6]{sample-#1.pdf}}{}}
% \isample{amc}
% \isample{separate}
% \isample{plain}
%
%
% \section{Usage}
%
% \subsection{Package options}\label{d:options}
% The following options are available for package \textsf{automultiplechoice}:
% \def\ioption#1{\item[\texttt{#1}]\SpecialOptIndex{#1}}
% \begin{itemize}
% \ioption{noshuffle} cancels answers shuffling for all questions.
%
% \ioption{answers} produces a common corrected answers sheet.
%
% \ioption{indivanswers} shows the boxes that corresponds to correct
% choices on the question sheet.
%
% \ioption{box} includes every question in a \LaTeX{} box, so that
% they can't be cutted on two different pages.
%
% \ioption{separateanswersheet} asks for a separate answer sheet (see
% section~\ref{s:separate} for an example). Commands |\AMCformBegin|
% and |\AMCform| must be used to describe the separate answer sheet
% (see section~\ref{d:separate}).
%
% \ioption{digits} puts digits instead of letters in the boxes, when
% |separateanswersheet| (or |insidebox|) is used.
%
% \ioption{outsidebox} prints boxes labels outside the boxes on the
% answersheet when |separateanswersheet| is set.
%
% \ioption{init} initializes the random generator from
% time. \emph{This option is only for testing: don't use it for a real
% exam!}
%
% \ioption{completemulti} adds an answer
% ``\makeatletter\AMC@loc@none\makeatother'' at the end of each
% multiple question (question with no, one or several correct
% answers), so as to make the difference between ``I don't know'' and
% ``I think none of the answers are correct''.
%
% \ioption{insidebox} puts a letter (or a digit if |digits| option is
% used) inside the boxes, even if |separateanswersheet| is not
% used. The |insidebox| option is implicitely called when using
% |separateanswersheet|: no need to call it then.
%
% \ioption{calibration} asks for logging positions of boxes and
% markers in the |.xy| file. Without this option, a \LaTeX{} run
% updates the document but not the |.xy| file.
%
% \ioption{nowatermark} calcels the ``DRAFT'' indications above pages.
%
% \ioption{catalog} is used for formatting a catalog of questions, not
% an exam. Then the questions identifiers will be printed.
%
% \ioption{francais} asks for french localisation.
%
% \ioption{lang=XX} asks for localisation in |XX| language. At
% present, only |DE| (German), |ES| (Spanish), |FR| (French), |IT|
% (Italian), |JA| (Japanese) and |NL| (Dutch) are available.
%
% \ioption{plain} cancels \textsf{environ} and \textsf{etex} automatic
% loading. The default behaviour is to load \textsf{environ} and
% \textsf{etex} packages if available, as they improve
% \textsf{automultiplechoice}. This is not done when |plain| option is set.
%
% \ioption{nopage} cancels markers print and page layout definition
% (see sample in section~\ref{s:plain}).
%
% \ioption{postcorrect} tells that correct answers won't be given in
% the LaTeX source. The teacher will fill one answer sheet for AMC to
% analyse the scan and set correct answers from it.
%
% \end{itemize}
%
% See also section~\ref{d:french} for a french version of some of
% these options.
%
% \subsection{Questions and answers}
%
% We make a difference between two kind of multiple choice questions:
% \begin{itemize}
% \item \textbf{Simple questions}: there is one and only one correct
%   choices among the proposed choices, \emph{and this is announced to
%   the student}. Thus, the student is asked to check one answer if he
%   thinks this is the good one, and to check none if he has no idea.
% \item \textbf{Multiple questions}: there can be zero, one or several
%   correct choices among the proposed choices. This is also announced
%   to the student (using the |\multiSymbole| sign, with
%   default~\multiSymbole{}), so that the student is asked to check all
%   the boxes corresponding to correct choices, and to let unchecked
%   all boxes corresponding to wrong choices.
% \end{itemize}
%
% \DescribeEnv{question}\DescribeEnv{questionmult}
% Simple questions are enclosed in a |{question}|\marg{id}
% environment, and multiple questions are enclosed in a
% |{questionmult}|\marg{id} environment. These environments contain
% the question text, and the proposed choices inside a |choices|-like
% environment (see next). The \meta{id} argument is a question
% identifier. Each question must have a unique identifier, different
% from the other questions identifiers.
%
% \vspace{2ex}
% \begin{code}
%   \begin{question}{everest}
%     What is the elevation of Mount Everest?
%     \begin{choices}
%       \correctchoice{8,848\,m}
%       \wrongchoice{8,253\,m}
%       \wrongchoice{8,810\,m}
%     \end{choices}
%   \end{question}
%
%   \begin{questionmult}{americas}
%     Which contries are in the Americas?
%     \begin{choices}
%       \correctchoice{Guatemala}
%       \correctchoice{Canada}
%       \wrongchoice{Switzerland}
%       \wrongchoice{Cambodia}
%     \end{choices}
%   \end{questionmult}
% \end{code}
% \vspace{2ex}
%
% \DescribeMacro{\AMCcompleteMulti}\DescribeMacro{\AMCnoCompleteMulti}
% For multiple questions, it is sometimes useful to make the
% difference between a student who thinks that none of he choices are
% correct, and a student who did not answer the question. The use of
% package option |completemulti| can be used in this case: it
% adds a choice to all multiple questions. Commands |\AMCcompleteMulti| and
% |\AMCnoCompleteMulti| can also be used to change this behaviour for a
% single question.
%
% \vspace{2ex}
% \begin{code}
%   \begin{questionmult}{americas}
%     \AMCcompleteMulti
%     Which contries are in the Americas?
%     \begin{choices}
%       \correctchoice{Guatemala}
%       \correctchoice{Canada}
%       \wrongchoice{Switzerland}
%       \wrongchoice{Cambodia}
%     \end{choices}
%   \end{questionmult}
% \end{code}
% \vspace{2ex}
%
% \DescribeEnv{choices}\DescribeEnv{choiceshoriz}\DescribeEnv{choicescustom}
% Depending on the formatting style for answers, one can choose one of
% the following ones:
% \begin{itemize}
% \item Environment |choices| is usualy chosen for long answers:
% \amccode
% \begin{code}
%  \begin{questionmult}{latex}
%    What are the possible uses of latex?
%    \begin{choices}
%      \correctchoice{Natural rubber is
%        the most important product
%        obtained from latex.}
%      \correctchoice{Latex from the chicle
%        and jelutong trees is used in
%        chewing gum.}
%      \wrongchoice{Latex is used as a fuel
%        for some space launch vehicles.}
%    \end{choices}
%  \end{questionmult}
% \end{code}
% \item environment |choiceshoriz| is chosen for short answers:
% \amccode
% \begin{code}
%  \begin{question}{insect}
%    From those animals, which
%    is an insect?
%    \begin{choiceshoriz}
%      \correctchoice{Ant}
%      \wrongchoice{Horse}
%      \wrongchoice{Turtle}
%    \end{choiceshoriz}
%  \end{question}
% \end{code}
%
% \item environment |choicescustom| is provided to customize answers
%   formatting. See~\ref{c:answers} for details.
% \end{itemize}
%
% \DescribeMacro{\correctchoice}\DescribeMacro{\wrongchoice} As you
% have seen in these examples, the |choices|-like environments contain
% |\correctchoice|\marg{text} and |\wrongchoice|\marg{text} commands,
% with the text of the proposed choice as argument.
%
% \subsection{Scoring}
%
% \DescribeMacro{\scoring}\DescribeMacro{\scoringDefaultM}
% \DescribeMacro{\scoringDefaultS}\DescribeMacro{\QuestionIndicative}
% Scoring strategies can be given in the \LaTeX{} source. They don't
% have any impact on the question sheet: they are only transmitted to
% the analysis software through the |.amc| file. See AMC documentation
% to write proper commands for your needs.  |\scoring|\marg{score} can
% be used inside a |question| or |questionmult| environment to
% describe the scoring strategy for the question, or after a
% |\correctchoice| or |\wrongchoice| command to describe score
% associated to a particular choice. |\scoringDefaultM|\marg{score}
% and |\scoringDefaultS|\marg{score} define default scoring strategies
% for multiple and simple questions. |\QuestionIndicative| tags a
% question that is not taken into account to compute the mark -- for
% example, it can be used for a question about the way students have
% enjoyed the course.
%
% \subsection{Groups of questions}\label{d:groups}
%
% Several commands are available that allows shuffling questions for
% each question sheet. They handle groups of questions.  These groups
% will usualy contain questions, but can be made of any \LaTeX{} content.
% 
% \DescribeMacro{\element}\DescribeMacro{\shufflegroup}\DescribeMacro{\insertgroup}
% The command |\element|\marg{groupname}\marg{content} adds element
% with content \meta{content} to the group named \meta{groupname}.
% The command |\shufflegroup|\marg{groupname} shuffles elements of
% group named \meta{groupname}. The command
% |\insertgroup|\oarg{n}\marg{groupname} inserts elements of group
% \meta{groupname} one after one. If optional parameter \meta{n} is
% given, only the first \meta{n} elements of the group are inserted in
% the document.
%
% As an example without questions in groups elements, consider the
% following code:
%
%\iffalse
%<*doc>
%\fi
\begin{verbatim}
\element{serie}{ one}
\element{serie}{ two}
\element{serie}{ three}
\element{serie}{ four}
\element{serie}{ five}
Numbers:\insertgroup{serie}.

\shufflegroup{serie}
Two of them:\insertgroup[2]{serie}.
\end{verbatim}
%\iffalse
%</doc>
%\fi
%   which produces:
%   \AMCrandomseed{1237893}
%   \element{serie}{ one}
%   \element{serie}{ two}
%   \element{serie}{ three}
%   \element{serie}{ four}
%   \element{serie}{ five}
%   \begin{center}\fbox{\begin{minipage}{.7\linewidth}
%       Numbers:\insertgroup{serie}.
%       
%       \shufflegroup{serie}
%       Two of them:\insertgroup[2]{serie}.
%   \end{minipage}}\end{center}
%
% \DescribeMacro{\cleargroup}\DescribeMacro{\copygroup} The command
% |\cleargroup|\marg{groupname} clears all the elements of group
% \meta{groupname}, making an empty group. The command
% |\copygroup|\oarg{n}\marg{from}\marg{to} copies the elements of
% group \meta{from} to grou \meta{to} -- if optional parameter
% \meta{n} is given, only the \meta{n} first elements are copied.
%
% As an example again without questions, consider the following
% code:
%\iffalse
%<*doc>
%\fi
\begin{verbatim}
\element{digits}{ 1}\element{digits}{ 2}\element{digits}{ 3}
\element{digits}{ 4}\element{digits}{ 5}\element{digits}{ 6}
\element{digits}{ 7}\element{digits}{ 8}\element{digits}{ 9}
\element{letters}{ A}\element{letters}{ B}\element{letters}{ C}
\element{letters}{ D}\element{letters}{ E}\element{letters}{ F}

\shufflegroup{digits}\shufflegroup{letters}
\cleargroup{mixed}
\copygroup[3]{digits}{mixed}\copygroup[2]{letters}{mixed}
\shufflegroup{mixed}
Three digits and two letters:\insertgroup{mixed}.

\shufflegroup{digits}\shufflegroup{letters}
\cleargroup{mixed}
\copygroup[3]{digits}{mixed}\copygroup[2]{letters}{mixed}
\shufflegroup{mixed}
Three digits and two letters:\insertgroup{mixed}.
\end{verbatim}
%\iffalse
%</doc>
%\fi
% which produces:
% \AMCrandomseed{1237899}
%     \element{digits}{ 1}
%     \element{digits}{ 2}
%     \element{digits}{ 3}
%     \element{digits}{ 4}
%     \element{digits}{ 5}
%     \element{digits}{ 6}
%     \element{digits}{ 7}
%     \element{digits}{ 8}
%     \element{digits}{ 9}
%     \element{letters}{ A}
%     \element{letters}{ B}
%     \element{letters}{ C}
%     \element{letters}{ D}
%     \element{letters}{ E}
%     \element{letters}{ F}
%     \begin{center}\fbox{\begin{minipage}{.7\linewidth}
%         \shufflegroup{digits}\shufflegroup{letters}
%         \cleargroup{mixed}
%         \copygroup[3]{digits}{mixed}\copygroup[2]{letters}{mixed}
%         \shufflegroup{mixed}
%         Three digits and two letters:\insertgroup{mixed}.
%
%         \shufflegroup{digits}\shufflegroup{letters}
%         \cleargroup{mixed}
%         \copygroup[3]{digits}{mixed}\copygroup[2]{letters}{mixed}
%         \shufflegroup{mixed}
%         Three digits and two letters:\insertgroup{mixed}.
%       \end{minipage}}\end{center}
%
%
% You can find an example involving questions in
% section~\ref{d:samples}.
%
% \subsection{Students identification}
%
% \DescribeMacro{\namefield}\DescribeMacro{\AMCcode}\DescribeMacro{\AMCcodeH}
% There are two ways to associate students to their sheets.
% \begin{itemize}
% \item Always add to one page of each copy some place for the student
%   to write down his name. If you want AMC software to be able to cut
%   the scan arount this area to present it to you and ask you to read
%   the written name (this is called manual association), you must use
%   the |\namefield|\marg{descr} command. The \meta{descr} argument
%   contains the \LaTeX{} code used to format the name field on the
%   page. For example:
% \vspace{2ex}
%   \begin{code}
%  \namefield{\fbox{    
%    \begin{minipage}{15em}
%      Name and surname:\vspace*{3ex}\par
%      \noindent\dotfill\vspace{2mm}
%    \end{minipage}
%  }}
%   \end{code}
% \vspace{2ex}
%   You can see that the |\namefield| command has no effect on the
%   produced document. In fact, its only purpose is to log in the
%   |.xy| file information about the position of the name field on the
%   page, to be used by the software analysing the scans. 
% \item For automated student identification, if for example students
%   have a 6-digits student number, you can ask them to code it
%   somewhere on the question sheet. This can be done using the
%   |\AMCcode|\marg{key}\marg{ndigits} command, where \meta{key} is the key
%   identifier, that can be used to retrieve coded student numbers
%   from the scans, and \meta{ndigits} is the number of digits for
%   numbers to be coded.
% \vspace{2ex}
%   \begin{code}
%  \AMCcode{student}{6}
%
%   \end{code}
%
%   For smaller number of digits, the ``horizontal'' form can be preferred: 
%   {\PexaDefaults{samplewidth=.7\PexaWidth}
% \vspace{2ex}
%   \begin{code}
%  \AMCcodeH{student}{3}  
%   \end{code}
%   }
% \end{itemize}
%
% \subsection{Separate answer sheet}\label{d:separate}
%
% \DescribeMacro{\AMCformBegin}\DescribeMacro{\AMCform}
% \DescribeMacro{\AMCcleardoublepage}
% To produce separate answer sheets as seen in section~\ref{s:separate},
% \begin{enumerate}
% \item use the
%   \SpecialOptIndex{separateanswersheet}|separateanswersheet| package
%   option.
% \item use the |\AMCformBegin| command at the beginning of the answer
%   sheet description. This command usualy follows a command to get a
%   new page. This command can be the classical |\clearpage| for
%   single-sided question sheets, or the |\AMCcleardoublepage|
%   command, that go to the next odd numbered page, so that the answer
%   sheet is on a separate sheet even when printing in duplex mode.
% \item use the |\AMCform| command to insert all boxes for all
%   questions.
% \end{enumerate}
%
% See section~\ref{s:separate} for an example.
%
% \subsection{Random computation questions}
%
% One can use the \LaTeX{} package \textsf{fp} to make random computation
% questions, as can be seen in the following example (don't forget to
% load package \textsf{fp}):
% \amccode
% {\makeatletter\AMC@correctrue\makeatother
% \begin{code}
%  \begin{question}{simplesum}
%    \FPeval\VQa{trunc(1+random*8,0)}
%    \FPeval\VQb{trunc(4+random*5,0)}
%    \FPeval\VQsum{clip(VQa+VQb)}
%    \FPeval\VQnoA{clip(VQa+VQb-1)}
%    \FPeval\VQnoB{clip(VQa*VQb)}
%    \FPeval\VQnoC{clip(VQa-VQb)}
%    How much are \VQa{} plus \VQb{}?
%    \begin{choiceshoriz}
%      \correctchoice{\VQsum}
%      \wrongchoice{\VQnoA}
%      \wrongchoice{\VQnoB}
%      \wrongchoice{\VQnoC}
%    \end{choiceshoriz}
%  \end{question}
% \end{code}
% }
% In this example, |\VQa| and |\VQb| are used to store two random
% integers (the first between 1 and 8, and the second between 4 and
% 8). Then |\VQsum| stores the sum of these two integers, and
% |\VQnoA|, |\VQnoB| and |\VQnoC| are other values that will be used
% as distractors in the multiple choice question.
%
% \DescribeMacro{\AMCIntervals}In some cases, command
% |\AMCIntervals|\marg{x}\marg{x0}\marg{x1}\marg{delta} from
% \textsf{automultiplechoice} can be useful. It adds a sequence of choices
% made of intervals $[x_i,x_i+\delta[$ of length \meta{delta} covering
% the interval $[\meta{x0},\meta{x1}[$, using |\correctchoice| when
% \meta{x} lies in the interval, and |\wrongchoice| otherwise.
%
%\iffalse
%<*doc>
%\fi
\begin{verbatim}
\begin{question}{inf-expo-indep}
  \FPeval\VQa{trunc(2 + random * 4,0)}
  \FPeval\VQb{trunc(6 + random * 5,0)}
  \FPeval\VQr{VQa/(VQa+VQb)}
  Let $X$ and $Y$ be two independent random variables, following
  exponential laws with respective parameters \VQa{} and \VQb{}.
  In which interval lies the probability $\textrm{P}[X<Y]$?
  \begin{multicols}{5}
    \begin{reponses}[o]
      \AMCIntervals{\VQr}{0}{1}{0.1}
    \end{reponses}
  \end{multicols}
\end{question}
\end{verbatim}
%\iffalse
%</doc>
%\fi
% {\makeatletter\AMC@correctrue\makeatother
% \fbox{
%  \begin{minipage}{.9\linewidth}
%  \begin{question}{inf-expo-indep}
%    \FPeval\VQa{trunc(2 + random * 4,0)}
%    \FPeval\VQb{trunc(6 + random * 5,0)}
%    \FPeval\VQr{VQa/(VQa+VQb)}
%
%    Let $X$ and $Y$ be two independent random variables, following
%    exponential laws with respective parameters \VQa{} and \VQb{}.
%    In which interval lies the probability $\textrm{P}[X<Y]$?
%    \begin{multicols}{5}
%      \begin{reponses}[o]
%        \AMCIntervals{\VQr}{0}{1}{0.1}
%      \end{reponses}
%    \end{multicols}
%  \end{question}
%  \vspace*{.1ex}
% \end{minipage}
% }}
%
%   \DescribeMacro{\AMCnumericChoices}One can also use the
%   |\AMCnumericChoices| command to ask the student to enter a
%   numerical value as his answer, as in the following example:
%\iffalse
%<*doc>
%\fi
\begin{verbatim}
  \begin{questionmultx}{sqrt}
    \FPeval\VQa{trunc(5+random*15,0)}
    \FPeval\VQs{VQa^0.5}

    Compute $\sqrt{\VQa}$ and round it with two digits after period.

    \AMCnumericChoices{\VQs}{digits=3,decimals=2,sign=true,
      borderwidth=0pt,backgroundcol=lightgray,approx=5}
  \end{questionmultx}
\end{verbatim}
%\iffalse
%</doc>
%\fi
% \begin{center}
% {\makeatletter\AMC@correctrue\makeatother
% \fbox{
%  \begin{minipage}{.7\linewidth}
%  \begin{questionmultx}{sqrt}
%    \FPeval\VQa{trunc(5+random*15,0)}
%    \FPeval\VQs{VQa^0.5}
%
%    Compute $\sqrt{\VQa}$ and round it up to two digits after period.
%
%    \AMCnumericChoices{\VQs}{digits=3,decimals=2,sign=true,
%      borderwidth=0pt,backgroundcol=lightgray,approx=5}
%  \end{questionmultx}
%  \end{minipage}
% }}
% \end{center}
%
% Note the use of |questionmultx| environment: we need this question
% to be \emph{multiple} as several boxes has to be ticked, but we
% can't say that \emph{several answers are correct}, so we don't show
% the \multiSymbole.
% 
%   Available options that can be used in the second argument of the
%   |\AMCnumericChoices| command are the following (\meta{bool} can
%   be |true| or |false|, and \meta{color} must be a color known by the |xcolor| package):
%   \begin{itemize}\setlength{\itemindent}{5em}
%   \item[|digits=|\meta{num}] gives the number of significant digits to request (defaults to 3).
%   \item[|decimals=|\meta{num}] gives the number of digits after
%     period to request (defaults to 0). Note that when |decimals| is
%     positive, the LaTeX package |fp| must be loaded.
%   \item[|base=|\meta{num}] gives the base for digits and decimals (defaults to 10).
%   \item[|sign=|\meta{bool}] requests (or not) a signed value (default to true).
%   \item[|strict=|\meta{bool}] if |true|, a box has to be ticked for
%     every digit and for the sign. If |false|, if some digits has no
%     ticked box, they will be set to zero. Defaults to |false|.
%   \item[|vertical=|\meta{bool}] if |true|, each digit is represented
%     on one raw. If |false| (default), each digit is represented on one
%     line.
%   \item[|reverse=|\meta{bool}] if |true|, place higher values of the
%     digits on the top in vertical mode (defaults to |true|).
%   \item[|vhead=|\meta{bool}] if |true|, in vertical mode, a header
%     is placed over all digits rows, made using the command
%     |\AMCntextVHead| that is originally defined as
%     |\def\AMCntextVHead#1{\emph{b#1}}|. This default value is
%     useful to number the binary digits. Default value is |false|.
%   \item[|hspace=|\meta{space}] sets the horizontal space between
%     boxes (defaults to |.5em|).
%   \item[|vspace=|\meta{space}] sets the certical space between
%     boxes (defaults to |1ex|).
%   \item[|borderwidth=|\meta{space}] sets the width of the frame
%     around all the boxes (defaults to |1mm|).
%   \item[|bordercol=|\meta{color}] sets the color of the frame
%     (defaults to |lightgray|).
%   \item[|backgroundcol=|\meta{color}] sets the background color
%     (defaults to |white|).
%   \item[|Tsign=|\meta{text}] sets the text to print at the top of the
%     boxes to set the sign (Can also be redefined by
%     |\def\AMCntextSign|\marg{text}, and defaults to be empty).
%   \item[|Tpoint=|\meta{text}] sets the text for the period. Can also
%     be redefined by |\def\AMCdecimalPoint|\marg{text}, and defaults
%     to |\raisebox{1ex}{\bf .}|.
%   \item[|scoring=|\meta{bool}] if |true|, a scoring strategy is given
%     to AMC for this question. Defaults to |true|.
%   \item[|scoreexact=|\meta{num}] gives the score for an exact answer (defaults to 2).
%   \item[|scoreapprox=|\meta{num}] gives the score for an approximative answer (defaults to 1).
%   \item[|approx=|\meta{num}] sets the maximal distance to the
%     correct integer value (value without the decimal point) for an
%     answer to be said \emph{approximative} and be rewarded to
%     |scoreapprox| points (defaults to 0).
%   \end{itemize}
%
%   The text added at the end of the questions using
%   |\AMCnumericChoices| when not in the separate answer sheet (and
%   when a separate answer sheet is requested by the
%   |separateanswersheet| package option) can also be set redefining
%   the |\AMCntextGoto| command, as:
%
%\iffalse
%<*doc>
%\fi
\begin{verbatim}
\def\AMCntextGoto{\par{\bf\emph{Please code the answer on
                           the separate answer sheet.}}}
\end{verbatim}
%\iffalse
%</doc>
%\fi
%   
% \subsection{French command names}\label{d:french}
% For backward compatibility, some of \textsf{automultiplechoice}
% commands, environments and package option have their French
% counterpart. You can always use either the English command or the
% French equivalent. See table~\ref{t:french} for details.
%
% \begin{table}[htb]\centering
% \SpecialUsageIndex{\champnom}
% \SpecialUsageIndex{\bonne}
% \SpecialUsageIndex{\mauvaise}
% \SpecialUsageIndex{\alafin}
% \SpecialUsageIndex{\choixIntervalles}
% \SpecialUsageIndex{\bareme}
% \SpecialUsageIndex{\baremeDefautM}
% \SpecialUsageIndex{\baremeDefautS}
% \SpecialUsageIndex{\exemplaire}
% \SpecialUsageIndex{\melangegroupe}
% \SpecialUsageIndex{\restituegroupe}
% \SpecialUsageIndex{\formulaire}
% \SpecialUsageIndex{\AMCdebutFormulaire}
% \SpecialEnvIndex{reponses}
% \SpecialEnvIndex{reponseshoriz}
% \SpecialEnvIndex{reponsesperso}
% \SpecialEnvIndex{copieexamen}
% \SpecialOptIndex{ordre}
% \SpecialOptIndex{correc}
% \SpecialOptIndex{correcindiv}
% \SpecialOptIndex{bloc}
% \SpecialOptIndex{ensemble}
% \SpecialOptIndex{chiffres}
% \begin{tabular}{\Vbar c\Vbar c\Vbar c\Vbar }
% \hline
%   type & English & French \\
% \hline
%   command & |\namefield| & |\champnom| \\
%   environment & |choices| & |reponses| \\
%   environment & |choiceshoriz| & |reponseshoriz| \\
%   environment & |choicescustom| & |reponsesperso| \\
%   command & |\correctchoice| & |\bonne| \\
%   command & |\wrongchoice| & |\mauvaise| \\
%   command & |\lastchoices| & |\alafin| \\
%   command & |\AMCIntervals| & |\choixIntervalles| \\
%   \hline
%   command & |\scoring| & |\bareme| \\
%   command & |\scoringDefaultM| & |\baremeDefautM| \\
%   command & |\scoringDefaultS| & |\baremeDefautS| \\
%   \hline
%   command & |\onecopy| & |\exemplaire| \\
%   environment & |examcopy| & |copieexamen| \\
%   \hline
%   command & |\shufflegroup| & |\melangegroupe| \\
%   command & |\insertgroup| & |\restituegroupe| \\
%   \hline
%   command & |\AMCform| & |\formulaire| \\
%   command & |\AMCformBegin| & |\AMCdebutFormulaire| \\
%   \hline
%   option & |noshuffle| & |ordre| \\
%   option & |answers| & |correc| \\
%   option & |indivanswers| & |correcindiv| \\
%   option & |box| & |bloc| \\
%   option & |separateanswersheet| & |ensemble| \\
%   option & |digits| & |chiffres| \\
%   \hline
% \end{tabular}
% \caption{French equivalent commands}\label{t:french}
% \end{table}
%
% \subsection{Customisation}
%
% \subsubsection{Boxes}
% \DescribeMacro{\AMCboxDimensions} The command
% |\AMCboxDimensions|\marg{dims} can be used to specify the dimensions
% of the boxes to be ticked. The argument \meta{dims} is a
% coma-separated list of \meta{key}|=|\meta{value} pairs, with the
% following possible \meta{key}s:
% \begin{itemize}
% \item[|size|] for the size of the edges of the boxes.
% \item[|down|] for the length the boxes are to be moved down.
% \item[|rule|] for the rule width.
% \end{itemize}
% Default values are |\AMCboxDimensions{size=2.5ex,down=.4ex,rule=.5pt}|
%
% \DescribeMacro{\AMCboxColor} Also note that you can set the color
% used for the boxes (only the box that are to be filled by the
% students and will be used for data capture) with the
% |\AMCboxColor|\marg{color} command. This allows to print the boxes
% with some color that won't disturb too much the data capture (for
% example red, but some light grey can also be considered). The
% \meta{color} parameter must be a color recognized by the |xcolor|
% package.\amccode
% \begin{code}
%   \AMCboxColor{red}
%   \begin{question}{sum}$2+2={}$
%   \begin{choiceshoriz}[o]
%     \wrongchoice{1}\correctchoice{4}\wrongchoice{10}
%   \end{choiceshoriz}
%   \end{question}
% \end{code}
%
% \subsubsection{Codes}
% One may adapt the codes rendering from |\AMCcode| to one's needs
% modifying the following lengths:
% \begin{itemize}
% \item |\AMCcodeHspace| is the amount of horizontal space between two
%   columns of digits,
% \item |\AMCcodeHspace| is the amount of vertical space between two
%   rows of digits,
% \item |\AMCcodeBoxSep| is the amount of space between the box and
%   the label, when the label is printed outside the box.
% \end{itemize}
% Default values are |\AMCcodeHspace=.5em|
% |\AMCcodeVspace=.5em| |\AMCcodeBoxSep=.1em|
% 
% \subsubsection{Answers}\label{c:answers}
% Environment |choicescustom| will make use of the three commands
%   |\AMCbeginAnswer| (before the first answer), |\AMCendAnswer|
%   (after the last answer) and |\AMCanswer|\marg{box}\marg{text} (for
%   each answer) to format the answers. Redefining them properly, some
%   different answers formatting can be achieved. However, this does
%   not seem to work with non-trivial settings...  \amccode
% \begin{code}
%  \begin{question}{add}
%    \def\AMCbeginAnswer{$\Big($}
%    \def\AMCendAnswer{$\Big)$}
%    \def\AMCanswer#1#2{#1  #2\hfill}
%    2+2=
%    \begin{choicescustom}
%      \correctchoice{4}
%      \wrongchoice{2}
%      \wrongchoice{3}
%    \end{choicescustom}
%  \end{question}
% \end{code}
%
% \section{Implementation}
%
% This package uses the following other packages:
%    \begin{macrocode}
\RequirePackage{xcolor} % \fcolorbox to fill (or not) a box
\RequirePackage{fancyhdr} % \pagestyle{empty}
\RequirePackage{bophook} % \AtBeginPage
\RequirePackage{xkeyval} % \setkeys
\RequirePackage{rotating} % \rotatebox
%    \end{macrocode}
% It defines a version string:
%    \begin{macrocode}
\def\AMC@VERSION{AMC 1.1.1 svn:1104 with style $Revision: 431 $}
%    \end{macrocode}
% \begin{macro}{\AMC@amclog}\begin{macro}{\AMCmessage}
%   Informations about questions and choices will be logged to a file
%   with extension |amc|, to be parsed later. Macro |\AMC@amclog|
%   writes to this file.
%    \begin{macrocode}
\newwrite\AMC@logfile
\immediate\openout\AMC@logfile=\jobname.amc
\def\AMC@amclog#1{\immediate\write\AMC@logfile{#1}}
\def\AMCmessage#1{\AMC@amclog{AUTOQCM[#1]^^J}}
%    \end{macrocode}\end{macro}\end{macro}
%
% \begin{macro}{\AMC@LR}
%   Colours management can be faulty in right-to-left mode: in these
%   situations, we will make use of |\LR| from package \textsf{bidi} to
%   get back to left-to-right mode. |\AMC@LR| is |\LR| if \textsf{bidi}
%   is loaded.
%    \begin{macrocode}
\AtBeginDocument{\@ifpackageloaded{bidi}{%
  \PackageInfo{automultiplechoice}{Package bidi loaded: using LR for boxes.}%
  \let\AMC@LR=\LR}%
{\let\AMC@LR=\relax}}%
%    \end{macrocode}
% \end{macro}
%
% \subsection{Variables}
%
% Counters and boolean variables defined here are internal and should
% not be modified by the user.
%
% The package defines the following counters:
% \begin{description}
% \item |\AMCload@counter| number of choices already loaded for
%   current question.
% \item |\AMCid@quest| current question ID number (see section~\ref{s:keyid}).
% \item |\AMCid@etud| current student sheet number.
% \item |\AMCid@etudstart| starting student sheet number of the
%   current |onecopy| bloc.
% \item |\AMCid@check| current page checking number.
% \item |\AMCid@etudfin| last student sheet number for the exam.
% \item |\AMCnum@copies| number of exam sheets to produce.
% \end{description}
%
% It also defines the following switches:
% \begin{description}
% \item |\ifAMC@ordre| if choices are never to be shuffled. 
% \item |\ifAMC@correchead| if some correction header is to be printed
%   at the beginning.
% \item |\ifAMC@affichekeys| if questions keys are to be printed.
% \item |\ifAMC@correc| if correct choices are to be checked on the
%   produced document.
% \item |\ifAMC@qbloc| if questions are to be included in \LaTeX{}
%   boxes (so that they can't be splitted on two different pages).
% \item |\ifAMC@rbloc| if answers are to be included in \LaTeX{} boxes
%   (so that they can't be splitted on two different columns for
%   example).
% \item |\ifAMCcomplete@multi| if a choice
%   ``\makeatletter\AMC@loc@none\makeatother'' is to be added to every
%   multiple question.
% \item |\ifAMC@calibration| if this \LaTeX{} run is used to get page layouts.
% \item |\ifAMC@plain| if \textsf{automultiplechoice} won't try to load
%   useful packages (\textsf{etex}, \textsf{environ}) that extend
%   \textsf{automultiplechoice} capabilities.
% \item |\ifAMCune@bonne| if there is at least one correct answer for the current question.
% \item |\ifAMCtype@multi| if the current question is a multiple question.
% \item |\ifAMC@watermark| if the document is a draft, not to be used for exam.
% \item |\ifAMC@ensemble| if answers are to be given on a separate
%   answers sheet.
% \item |\ifAMC@inside@box| if a letter or digit is to be printed
%   inside all boxes.
% \item |\ifAMC@inside@digit| if digits are to be written inside
%   boxes instead of letters (when using a separate answer sheet for
%   example).
% \item |\ifAMC@outside@box| if labels for boxes are to be printed outside
%   the box on the answer sheet.
% \item |\ifAMCformulaire@dedans| is true for questions inside
%   separate answer sheet.
% \item |\ifAMC@zoneformulaire| is true for codes (made by |\AMCcode|)
%   inside separate answer sheet.
% \item |\ifAMC@pagelayout| is true if the AMC page layout, with signs
%   for scan analysis, is to be used.
% \item |\ifAMC@postcorrect| corresponds to the use of the
%   |postcorrect| package option.
% \end{description}
%    \begin{macrocode}
\newcount\AMCload@counter
\newcount\AMCid@quest\AMCid@quest=-1
\newcount\AMCid@check
\newcount\AMCid@etud\AMCid@etud=0
\newcount\AMCid@etudstart\AMCid@etudstart=0
\newcount\AMCid@etudfin
\newcount\AMCnum@copies
%    \end{macrocode}
%    \begin{macrocode}
\newif\ifAMC@ordre\AMC@ordrefalse
\newif\ifAMC@correchead\AMC@correcheadfalse
\newif\ifAMC@affichekeys\AMC@affichekeysfalse
\newif\ifAMC@correc\AMC@correcfalse
\newif\ifAMC@qbloc\AMC@qblocfalse
\newif\ifAMC@rbloc\AMC@rblocfalse
\newif\ifAMCcomplete@multi\AMCcomplete@multifalse
\newif\ifAMC@calibration\AMC@calibrationfalse
\newif\ifAMC@plain\AMC@plainfalse
\newif\ifAMCune@bonne
\newif\ifAMCtype@multi
\newif\ifAMC@watermark\AMC@watermarktrue
\newif\ifAMC@inside@box\AMC@inside@boxfalse
\newif\ifAMC@outside@box\AMC@outside@boxfalse
\newif\ifAMC@ensemble\AMC@ensemblefalse
\newif\ifAMC@inside@digit\AMC@inside@digitfalse
\newif\ifAMCformulaire@dedans\AMCformulaire@dedansfalse
\newif\ifAMC@zoneformulaire
\newif\ifAMC@pagelayout\AMC@pagelayouttrue
\newif\ifAMC@postcorrect\AMC@postcorrectfalse
\let\AMCcompleteMulti=\AMCcomplete@multitrue
\let\AMCnoCompleteMulti=\AMCcomplete@multifalse
%    \end{macrocode}
%
% \begin{macro}{\AMCid@name}
%   The package also defines command |\AMCid@name| to be the current
%   question identifier key.
%    \begin{macrocode}
\def\AMCid@name{}
%    \end{macrocode}\end{macro}
%
% \subsection{Dimensions}
%
% \begin{macro}{\AMCformVSpace}\begin{macro}{\AMCformHSpace}
% \begin{macro}{\AMCinterIrep}\begin{macro}{\AMCinterBrep}
% The following dimensions can be modified by the user to adjust
% questions formatting:
% \begin{description}
% \item |\AMCformVSpace| is the amount of vertical space between two
%   questions in a separate answer sheet.
% \item |\AMCformHSpace| is the amount of horizontal space between two
%   answers boxes in a separate answer sheet.
% \item |\AMCinterIrep| is the amount of vertical space to be added
%   between two answers.
% \item |\AMCinterBrep| is the amount of vertical space between two
%   boxed answers (see |\AMCBoxedAnswers| and |\ifAMC@rbloc|).
% \end{description}
%    \begin{macrocode}
\newdimen\AMCformVSpace\AMCformVSpace=1.2ex
\newdimen\AMCformHSpace\AMCformHSpace=.3em
\newdimen\AMCinterIrep\AMCinterIrep=\z@
\newdimen\AMCinterBrep\AMCinterBrep=.5ex
%    \end{macrocode}\end{macro}\end{macro}\end{macro}\end{macro}
%
% \subsection{Human readable sheet ID position}
%
% \begin{macro}{\AMCidsPosition}
% The position of the human readable sheet ID, near the corresponding
% binary boxes, is set with the |\AMCidsPosition| command, in the form
% |\AMCidsPosition{pos=|\meta{position}|,width=|\meta{width}|,height=|\meta{height}|}|,
% where \meta{position} is one of |side| (default), |top| and |none|,
% \meta{width} is the width of the box enclosing the ID (default value is
% |4cm|), and \meta{height} is the height of the box enclosing the ID
% (default value is |3ex|).
%
%    \begin{macrocode}
\newif\ifAMCids@top
\newif\ifAMCids@side
\newdimen\AMCids@width
\newdimen\AMCids@height
\define@choicekey*{AMCids}{pos}[\var\nr]{none,top,side}{%
  \ifcase\nr\relax
    \AMCids@topfalse\AMCids@sidefalse
  \or
    \AMCids@toptrue\AMCids@sidefalse
  \or
    \AMCids@topfalse\AMCids@sidetrue
  \fi
}
\define@key{AMCids}{width}{\AMCids@width=#1}
\define@key{AMCids}{height}{\AMCids@height=#1}
\def\AMCidsPosition#1{\setkeys{AMCids}{#1}}
\AMCidsPosition{pos=side,width=4cm,height=3ex}
%    \end{macrocode}\end{macro}
% 
% \subsection{Localisation}
%
% In this section, some localised strings or commands are defined, for
% English, French ans Spanish languages.
% \begin{macro}{\AMCtext}
%   To modify these texts, you can use command |\AMCtext|. For
%   example, |\AMCtext{draft}|\marg{text} sets the text to be printed
%   behind each page of a draft exam.
%    \begin{macrocode}
\def\AMCtext#1#2{\expandafter\def\csname AMC@loc@#1\endcsname{#2}}
%    \end{macrocode}\end{macro}
% \subsubsection{English}
% Text indicating draft exams:
%    \begin{macrocode}
\def\AMC@loc@draft{DRAFT}
%    \end{macrocode}
% Message at page bottom when compiled out of AMC gui:
%    \begin{macrocode}
\def\AMC@loc@message{For your examination, preferably print
  documents compiled from auto-multiple-choice.}
%    \end{macrocode}
% Annoucing a question in a separate sheet (parameter |#1| is the
% question number):
%    \begin{macrocode}
\def\AMC@loc@qf#1{\textbf{Question #1:}}
%    \end{macrocode}
% Annoucing a question (parameter |#1| is the question number and
% pamareter |#2| can be the multiple question symbol, or be empty):
%    \begin{macrocode}
\def\AMC@loc@q#1#2{\textbf{Question #1} #2}
%    \end{macrocode}
% Headers for corrected version and catalog:
%    \begin{macrocode}
\def\AMC@loc@corrected{Corrected}
\def\AMC@loc@catalog{Catalog}
%    \end{macrocode}
% Last choice added at the end for multiple questions when option
% |completemulti| is used:
%    \begin{macrocode}
\def\AMC@loc@none{None of these answers are correct.}
%    \end{macrocode}
%
% \subsubsection{Dutch}
%
% Dutch localisation is called with option |lang=NL|.
%    \begin{macrocode}
\def\AMC@loc@NL{
  \def\AMC@loc@draft{Ontwerp}
  \def\AMC@loc@message{Gebruik bij uw proefwerk bij voorkeur die
    documenten welke door auto-multiple-choice zijn aangemaakt.}
  \def\AMC@loc@qf##1{\textbf{Vraag ##1 :}}
  \def\AMC@loc@q##1##2{\textbf{Vraag ##1} ##2}
  \def\AMC@loc@corrected{Correctie}
  \def\AMC@loc@catalog{Catalogus}
  \def\AMC@loc@none{Geen van de antwoorden is juist.}
}
%    \end{macrocode}
%
% \subsubsection{French}
%
% French localisation is called with option |francais|, or |lang=FR|.
%    \begin{macrocode}
\def\AMC@loc@FR{
  \def\AMC@loc@draft{PROJET}
  \def\AMC@loc@message{Pour votre examen, imprimez de pr\'ef\'erence
    les documents compil\'es \`a l'aide de auto-multiple-choice.}
  \def\AMC@loc@qf##1{\textbf{Question ##1 :}}
  \def\AMC@loc@q##1##2{\textbf{Question ##1} ##2}
  \def\AMC@loc@corrected{Correction}
  \def\AMC@loc@catalog{Catalogue}
  \def\AMC@loc@none{Aucune de ces r\'eponses n'est correcte.}
}
%    \end{macrocode}
%
% \subsubsection{German}
%
% German localisation is called with option |lang=DE|.
%    \begin{macrocode}
\def\AMC@loc@DE{
  \def\AMC@loc@draft{ENTWURF}
  \def\AMC@loc@message{Benutzen Sie f\"ur Ihre Pr\"ufung bevorzugt Dokumente die mit
    auto-multiple-choice erstellt wurden.}
  \def\AMC@loc@qf##1{\textbf{Frage ##1 :}}
  \def\AMC@loc@q##1##2{\textbf{Frage ##1} ##2}
  \def\AMC@loc@corrected{Korrektur}
  \def\AMC@loc@catalog{Katalog}
  \def\AMC@loc@none{Keine dieser Antworten ist korrekt.}
}
%    \end{macrocode}
% 
% \subsubsection{Italian}
%
% Italian localisation is called with option |lang=IT|.
%    \begin{macrocode}
\def\AMC@loc@IT{
  \def\AMC@loc@draft{BOZZA}
  \def\AMC@loc@message{Per l'esame, \`e preferibile stampare i documenti
     a partire da auto-multiple-choice.}
  \def\AMC@loc@qf##1{\textbf{Domanda ##1:}}
  \def\AMC@loc@q##1##2{\textbf{Domanda ##1} ##2}
  \def\AMC@loc@corrected{Correzione}
  \def\AMC@loc@catalog{Catalogo}
  \def\AMC@loc@none{Nessuna risposta \`e giusta.}
}
%    \end{macrocode}
% 
% \subsubsection{Spanish}
%
% Spanish localisation is called with option |lang=ES|.
%    \begin{macrocode}
\def\AMC@loc@ES{
  \def\AMC@loc@draft{BORRADOR}
  \def\AMC@loc@message{Para revisi\'on, preferentemente imprimir documentos compilados
    desde auto-multiple-choice.}
  \def\AMC@loc@qf##1{\textbf{Pregunta ##1 :}}
  \def\AMC@loc@q##1##2{\textbf{Pregunta ##1} ##2}
  \def\AMC@loc@corrected{Correcci\'on}
  \def\AMC@loc@catalog{Cat\'alogo}
  \def\AMC@loc@none{Ninguna de estas preguntas son correctas.}
}
%    \end{macrocode}
% 
% \subsubsection{Japanese}
%
% Japanese localisation is called with option |lang=JA|. It includes UTF8 encoded
% Japanese characters which are shown as $\diamond$ here (look at the |.sty| file to see them).
%    \begin{macrocode}
\def\AMC@loc@JA{
  \def\AMC@loc@draft{ドラフト}
  \def\AMC@loc@message{試験の実施には、auto-multiple-choiceでコンパイルされた文書を印刷してください。}
  \def\AMC@loc@qf##1{\textbf{問##1:}}
  \def\AMC@loc@q##1##2{\textbf{問##1} ##2}
  \def\AMC@loc@corrected{模範解答}
  \def\AMC@loc@catalog{問題カタログ}
  \def\AMC@loc@none{該当なし。}
}
%    \end{macrocode}
% 
% \subsubsection{Other languages}
%
% Other languages can be integrated to \textsf{automultiplechoice}
% package upon request to the author.
%
% \subsection{Random}
% \subsubsection{Random pseudo-generator}
% The package uses the pseudo-random bit generator from
% \emph{TuGBoat} 1994, vol~15:1:
%    \begin{macrocode}
\ifx\AMC@SR\undefined\newcount\AMC@SR\fi
\providecommand\AMC@SRconst{2097152}
\providecommand\AMC@SRset[1]{\global\AMC@SR#1 \ignorespaces}
\providecommand\AMC@SRadvance{%
  \begingroup%
    \ifnum\AMC@SR<\AMC@SRconst\relax\AMC@SR@count\z@\else\AMC@SR@count\@ne\fi%
    \ifodd\AMC@SR\advance\AMC@SR@count\@ne\fi%
    \global\divide\AMC@SR\tw@%
    \ifodd\AMC@SR@count\global\advance\AMC@SR\AMC@SRconst\relax\fi%
  \endgroup}
\providecommand\AMC@SRbit{\AMC@SRadvance\ifodd\AMC@SR1\else0\fi}
\providecommand\AMC@SRtest[2]{\AMC@SRadvance%
  \ifodd\AMC@SR#2\else#1\fi\ignorespaces}
\providecommand\AMC@SRvalue{\number\AMC@SR}
%    \end{macrocode}
% \begin{macro}{\AMCrandomseed}
%   The seed of this generator is set to 1515, but another value can
%   be given using the command |\AMCrandomseed|\marg{seed}.
%    \begin{macrocode}
\AMC@SRset{1515}
\def\AMCrandomseed#1{\AMC@SRset{#1}}
%    \end{macrocode}\end{macro}
% \subsubsection{Uniform random deviates}
% \begin{macro}{\AMC@SRnextByte}\begin{macro}{\AMC@SRmax}
%     This generator is used to build first a 20-bit uniform integer
%     generator (macro |\AMC@SRnextByte|). Then, using modulo, a
%     (nearly) uniform generator on $\{0,\ldots,n-1\}$ is built:
%     command |\AMC@SRmax{|$n$|}| puts in |\AMC@SR@count| the random
%     deviate.
%    \begin{macrocode}
\newcount\AMC@SR@count
\def\AMC@SR@time{\AMC@SRset{\time}}
\newcount\AMC@SRnum
\def\AMC@SRnextByte{\AMC@SRnum=\z@%
  \AMC@SR@count=20%
  \loop\multiply\AMC@SRnum\tw@%
     \AMC@SRtest{\advance\AMC@SRnum\@ne}{}%
  \ifnum\AMC@SR@count>\@ne\advance\AMC@SR@count\m@ne\repeat%
}
\newcommand\AMC@SRmax[1]{\AMC@SRnextByte%
  \AMC@SR@count=\AMC@SRnum%
  \divide\AMC@SR@count by #1\relax%
  \multiply\AMC@SR@count by #1\relax%
  \advance\AMC@SRnum by -\AMC@SR@count%
}
%    \end{macrocode}\end{macro}\end{macro}
% \subsubsection{Tokens shuffling}
% \begin{macro}{\AMCsw@p}\begin{macro}{\AMC@shuffletoks}
%     The package defines the macro |\AMCsw@p| to swap the values of
%     two token registers given as parameters.
%
%     After defining $n$ token registers |\foo@i|, |\foo@ii|,
%     |\foo@iii|, |\foo@iv| and so on, you can shuffle them using
%     |\AMC@shuffletoks{|$n$|}{foo@}|.
%    \begin{macrocode}
\newcount\AMC@sti
\newtoks\AMCsw@p@
\newcommand\AMCsw@p[2]{%
  \global\AMCsw@p@=#1%
  \global#1=#2%
  \global#2=\AMCsw@p@}
\newcommand\AMC@shuffletoks[2]{%
  \AMC@sti=#1\relax%
  \@whilenum\AMC@sti>\@ne\do{%
    \AMC@SRmax{\AMC@sti}\advance\AMC@SRnum\@ne\relax%
    \AMCsw@p{\csname #2\romannumeral\AMC@SRnum\endcsname}%
            {\csname #2\romannumeral\AMC@sti\endcsname}%
    \advance\AMC@sti\m@ne\relax%
  }}
%    \end{macrocode}\end{macro}\end{macro}
%
% \subsection{Keys numbering}\label{s:keyid}
%
% \begin{macro}{\AMC@unnumero}\begin{macro}{\AMC@affecte}
%   This package allocates a unique integer ID to each question key
%   from the questionnary. The counter |\AMC@numerotation| keeps track
%   of the number of keys which already had an ID. Command
%   |\AMC@definitnumero{|$n$|}{key}| allocates ID $n$ to the key
%   |key|.  Command |\AMC@prepare{key}| looks if an ID had already
%   been associated to |key|, and, if not, makes a new ID allocation
%   for |key|.  Command |\AMC@unnumero{key}| returns the ID associated
%   with |key| (creating one if necessary). Command
%   |\AMC@affecte{key}{\cnt}| give to counter |\cnt| the value of the
%   ID associated to |key| (creating one if necessary).
%   \begin{macrocode}
\newcount\AMC@numerotation\AMC@numerotation=\z@%
\def\AMC@definitnumero#1#2{\AMC@amclog{AUTOQCM[NUM=#1=#2]^^J}%
  \expandafter\global\expandafter\def\csname AMC@numtab@#2\endcsname{#1}}
\def\AMC@prepare#1{\expandafter\ifx\csname AMC@numtab@#1\endcsname\relax%
  \global\advance\AMC@numerotation\@ne%
  \expandafter\AMC@definitnumero\expandafter{\the\AMC@numerotation}{#1}\fi}
\def\AMC@unnumero#1{\AMC@prepare{#1}\csname AMC@numtab@#1\endcsname}
\def\AMC@affecte#1#2{\AMC@prepare{#1}\global#2=\csname AMC@numtab@#1\endcsname}
%    \end{macrocode}\end{macro}\end{macro}
%
% \subsection{Boxes}
% \subsubsection{Position logging}
%
% \begin{macro}{\AMC@tracebox}\begin{macro}{\AMC@pagepos}
%     Command |\AMC@tracebox|\marg{trace}\marg{key}\marg{content} makes a
%     \LaTeX{} box around \meta{content}, and, if \meta{trace} is not
%     empty, logs to the |.xy| file informations to be able to compute
%     exact location of this box on the page, attached to the box
%     identification \meta{key}.
%
%   Command |\AMC@pagepos| logs page and page size informations at the
%   beginning of each page.
%
%    \begin{macrocode}
\def\AMC@tracepos#1#2{%
  \ifAMC@calibration\ifx\@empty#1\@empty\else%
  \pdfsavepos\protected@write\AMC@XYFILE{}{%
    \string\tracepos%
    {\the\AMCid@etud/\thepage:#2}%
    {\noexpand\number\pdflastxpos sp}%
    {\noexpand\number\pdflastypos sp}}%
  \fi\fi}
\def\AMC@traceposx#1#2{%
  \ifAMC@calibration\ifx\@empty#1\@empty\else%
  \pdfsavepos\protected@write\AMC@XYFILE{}{%
    \string\tracepos%
    {\the\AMCid@etud/\thepage:#2}%
    {\noexpand\number\pdflastxpos sp}%
    {0sp}}%
  \fi\fi}
\def\AMC@traceposy#1#2{%
  \ifAMC@calibration\ifx\@empty#1\@empty\else%
  \pdfsavepos\protected@write\AMC@XYFILE{}{%
    \string\tracepos%
    {\the\AMCid@etud/\thepage:#2}%
    {0sp}%
    {\noexpand\number\pdflastypos sp}}%
\fi\fi}
\newcommand\AMC@tracebox[3]{%
  \vbox{\AMC@traceposy{#1}{#2}%
    \hbox{\AMC@traceposx{#1}{#2}#3\AMC@traceposx{#1}{#2}}%
    \AMC@traceposy{#1}{#2}}}
\def\AMC@pagepos{%
  \ifAMC@calibration\protected@write\AMC@XYFILE{}{%
    \string\page%
    {\the\AMCid@etud/\thepage/\the\AMCid@check}%
    {\the\paperwidth}{\the\paperheight}}\fi}
%    \end{macrocode}\end{macro}\end{macro}
%
% \begin{macro}{\AMCdontScan}\begin{macro}{\AMCdontAnnotate}
%     The commands |\AMCdontScan| and |\AMCdontAnnotate| write
%     into the |xy| file instructions related to the current question.
%    \begin{macrocode}
\newcommand{\AMCdontScan}{\ifAMC@calibration\immediate\write\AMC@XYFILE{\string\dontscan{\the\AMCid@etud,\the\AMCid@quest}}\fi}
\newcommand{\AMCdontAnnotate}{\ifAMC@calibration\immediate\write\AMC@XYFILE{\string\dontannotate{\the\AMCid@etud,\the\AMCid@quest}}\fi}
%%    \end{macrocode}
% \end{macro}
% \end{macro}
% 
% \begin{environment}{amcxyfile}
%   The following lines defines an environment to use a particular file
%   for positions outputs. This is used mainly for documentation or testing.
%
%    \begin{macrocode}
\newwrite\AMC@XYspecial
\newwrite\AMC@tmpXY
\newenvironment{amcxyfile}[1]{%
  \openout\AMC@XYspecial#1%
  \let\AMC@tmpXY=\AMC@XYFILE%
  \let\AMC@XYFILE=\AMC@XYspecial%
}{\let\AMC@XYFILE=\AMC@tmpXY\closeout\AMC@XYspecial}
%    \end{macrocode}\end{environment}
%
% \begin{macro}{\namefield}
% The |\namefield|\marg{name field content} is a simple call to |\AMC@tracebox|:
%    \begin{macrocode}
\newcommand{\namefield}[1]{\AMC@tracebox{1}{nom}{#1}}
%    \end{macrocode}\end{macro}
% It is used to enclose the page region where students are to write
% their names, so as te retreive it easily from the scans. For example,
%\iffalse
%<*doc>
%\fi
\begin{verbatim}
\namefield{\fbox{%
  \begin{minipage}{5cm}
    Name:

    \vspace*{.5cm}\noindent\dotfill
    \vspace{2mm}
  \end{minipage}}}
\end{verbatim}
%\iffalse
%</doc>
%\fi
% produces the following box:
% \begin{amcxyfile}{automultiplechoice.xy2}
%   \begin{center}
%     \namefield{\fbox{%
%     \begin{minipage}{5cm}
%       Name:
%       
%       \vspace*{.5cm}\noindent\dotfill
%       \vspace{2mm}
%     \end{minipage}}}
%   \end{center}
% \end{amcxyfile}
% and outputs information about the position of the box in the |.xy|
% file, as seen in section~\ref{a:name}.
%
% \subsubsection{Boxes to be checked by students}
%
% \begin{macro}{\AMC@boxedchar}\begin{macro}{\AMCboxColor}
%   There are two styles for boxes to be checked by the students. The
%   first one is an empty box, printed beside the answer. The sencond
%   is a box with a character in it. It is mainly used when answers
%   are to be given on a separate answer sheet.
%
% These boxes can be drawn using command
% |\AMC@boxedchar|\marg{char}\marg{trace}\marg{key}\marg{filled}:
% \meta{char} is the character to print inside the box, \meta{trace}
% is non-empty if you want to log the box position in the |.xy| file,
% \meta{key} is the box identification, and \meta{filled} is non-empty
% for filling the box.
%
% The color of the boxes and of the characters drawn in them can be
% set using the command |\AMCboxColor|\marg{color}, where \meta{color}
% is a color that should be defined for the |xcolor| package. This
% color is used only in the case the box will be used for data
% capture: it is not used on the corrected answer sheet (|answers| or
% |indivanswers| package option), and not used on the subject part of
% an exam with a separate answer sheet (|separateanswersheet| package
% option).
%
% \begin{amcxyfile}{automultiplechoice.xy1}
% For example, |\AMC@boxedchar{K}{1}{test}{}| produce the box
% \makeatletter\AMC@boxedchar{K}{1}{test}{}\makeatother, writing the
% lines in the |.xy| file shown in section~\ref{a:boxed}.
% \end{amcxyfile}
%
%    \begin{macrocode}
\def\AMC@boxcolor{black}
\newcommand\AMCboxColor[1]{%
  \def\AMC@boxcolor{#1}}
\newcommand\AMC@boxedchar[4]{\AMC@LR{\hspace{0pt}%
  {\ifx\@empty#2\@empty \def\AMC@boxcolor{black}\fi%
   \ifAMC@correc\def\AMC@boxcolor{black}\fi%
   \fboxsep=\z@\fboxrule=\AMC@boxedrule%
    \lower\AMC@boxeddown\hbox{\fcolorbox{\AMC@boxcolor}%
      {\ifx\@empty#4\@empty white\else\AMC@boxcolor\fi}%
      {\vbox to \AMC@boxeddim{\AMC@tracepos{#2}{#3}\vfill %
          \hbox to \AMC@boxeddim{\hfill{\textcolor{\AMC@boxcolor}{#1}}\hfill}\vfill}%
        \AMC@tracepos{#2}{#3}}}}}%
}
%    \end{macrocode}\end{macro}\end{macro}
%
% \begin{macro}{\AMC@caselettre}\begin{macro}{\AMCchoiceLabel}
% Command |\AMC@caselettre| is the same as |\AMC@boxedchar|, but if
% \meta{char} is empty, it is replaced by an arabic or alphabetical
% counter, depending on the use of the |digits| package option.
%
% To use another way to label the choices boxes, the user can redefine
% the |\AMCchoiceLabel| macro, which takes as argument the name of the
% counter used to number the choices. One can for example use
% |\def\AMCchoiceLabel#1{\alph{#1}}| to ask for lowercase letters.
%
%    \begin{macrocode}
\def\AMCchoiceLabel#1{%
  \ifAMC@inside@digit\arabic{#1}%
  \else\Alph{#1}\fi%
}
\newcounter{AMC@ncase}
\setcounter{AMC@ncase}{0}
\newcommand\AMC@caselettre[4]{%
  \AMC@boxedchar{\ifx\@empty#1\@empty%
    \AMCchoiceLabel{AMC@ncase}%
    \else #1\fi}{#2}{#3}{#4}}
%    \end{macrocode}\end{macro}\end{macro}
% 
% \begin{macro}{\AMCboxDimensions}
%   The dimensions of these box are managed by
%   |\AMCboxDimensions|\marg{sizes}, where \meta{sizes} is a coma
%   separated list of \meta{name}|=|\meta{dimension} constructs. Here,
%   \meta{name} can be |size| for the box size, |rule| for the box rule
%   width and |down| for moving the box down.
%    \begin{macrocode}
\newlength\AMC@boxedrule
\newlength\AMC@boxeddown
\newlength\AMC@boxeddim
\define@key{AMCdim}{size}{\AMC@boxeddim=#1}
\define@key{AMCdim}{rule}{\AMC@boxedrule=#1}
\define@key{AMCdim}{down}{\AMC@boxeddown=#1}
\def\AMCboxDimensions#1{\setkeys{AMCdim}{#1}}
\AMCboxDimensions{size=2.5ex,down=.4ex,rule=.5pt}
%    \end{macrocode}\end{macro}
%
% \begin{macro}{\AMC@marque}\begin{macro}{\AMCformBox}\begin{macro}{\AMC@formBox}
%   Command |\AMC@marque|\marg{char}\marg{filled} prints a box with
%   character \meta{char} inside, and filled if \meta{filled} is
%   non-empty, using global variables to identify the box (question
%   and choice).
%
%   It calls
%   |\AMCformBox|\marg{char}\marg{filled}\marg{trace}\marg{key} to
%   actually render the box. This last command can be customized by
%   the user.
%
%   Command |\AMC@formBox| simply sets the first argument when empty
%   before calling |\AMCformBox|.
%
%    \begin{macrocode}
\newcommand\AMCformBox[4]{%
  \ifAMC@outside@box% letter to be written outside the box
    \AMC@boxedchar{}{#3}{#4}{#2}\hspace{.1em}{\textbf{#1}}%
  \else%
    \AMC@boxedchar{#1}{#3}{#4}{#2}%
  \fi%
}
\newcommand\AMC@formBox[4]{%
  \AMCformBox{\ifx\@empty#1\@empty%
    \AMCchoiceLabel{AMC@ncase}%
    \else #1\fi}{#2}{#3}{#4}%
}
\newcommand{\AMC@marque}[2]{%
  \ifAMC@ensemble%
    \ifAMC@zoneformulaire% for codes inside form sheet
      \protect\AMC@formBox{#1}{#2}{1}{case:\AMCid@name:\the\AMCid@quest,\the\AMCrep@count}%
    \else%
      \ifAMCformulaire@dedans% for answer boxes inside form sheet
        \protect\AMC@formBox{#1}{#2}{1}{case:\AMCid@name:\the\AMCid@quest,\the\AMCrep@count}%
      \else% outside form sheet: not to be read during data capture
        \AMC@caselettre{#1}{}{}{#2}%
    \fi\fi%
  \else% no separate sheet for answers: always read
    \ifAMC@inside@box%
      \AMC@caselettre{#1}{1}{case:\AMCid@name:\the\AMCid@quest,\the\AMCrep@count}{#2}%
    \else%
      \AMC@boxedchar{}{1}{case:\AMCid@name:\the\AMCid@quest,\the\AMCrep@count}{#2}%
    \fi%
  \fi%
}
%    \end{macrocode}\end{macro}\end{macro}\end{macro}
%
% \subsubsection{Binary boxes}
% 
% The package prints on each page some boxes that code (like binary
% digits) student sheet number, page number and a check number, so as
% to be read easily from scans after exam.
%
% \begin{macro}{\AMCid@checkmax}\begin{macro}{\AMC@NCBetud}
% \begin{macro}{\AMC@NCBpage}\begin{macro}{\AMC@NCBcheck}
%     The check number is just decreased each page. Its maximum value
%     is |\AMCid@checkmax|. The number of binary digits used to print
%     student sheet number, page and check number are |\AMC@NCBetud|,
%     |\AMC@NCBpage| and |\AMC@NCBcheck|. The number of the first page
%     is |\AMC@premierecopie|.
%     
%     The length of zone reserved for binary boxes is |\AMC@CBtaille|.
%    \begin{macrocode}
\def\AMCid@checkmax{60}
\def\AMC@NCBetud{12}
\def\AMC@NCBpage{6}
\def\AMC@NCBcheck{6}
\newlength{\AMC@CBtaille}\setlength{\AMC@CBtaille}{5cm}
\def\AMC@premierecopie{1}
%    \end{macrocode}
% \end{macro}\end{macro}\end{macro}\end{macro}
%
% \begin{macro}{\AMC@binaryBoxes}
%   Command |\AMC@binaryBoxes|\oarg{ndigits}\marg{n} prints \meta{ndigits}
%   boxes to represent number \meta{n} in its binary
%   form. |\AMCbin@one| and |\AMCbin@zero| print individual
%   digit-boxes.
%
%   For example, |\AMC@binaryBoxes[12]{367}| shows $367=000101101111_2$
%   using 12 boxes:
%   \begin{center}
%     \makeatletter\AMC@binaryBoxes[12]{367}\makeatother
%   \end{center}
%
%    \begin{macrocode}
\newtoks\AMCbin@sequence
\newcount\AMCbin@number
\newcount\AMCbin@ndigits
\newcount\AMCbin@id
\newcount\AMCbin@digit
\def\AMCbin@one{\advance\AMCbin@digit\@ne%
  \AMC@boxedchar{}{1}{chiffre:\the\AMCbin@id,\the\AMCbin@digit}{1}}
\def\AMCbin@zero{\advance\AMCbin@digit\@ne%
  \AMC@boxedchar{}{1}{chiffre:\the\AMCbin@id,\the\AMCbin@digit}{}}
\def\AMCbin@begin#1{\AMCbin@id=#1\AMCbin@digit=\z@}
\newcommand{\AMC@binaryBoxes}[2][1]{%
{\AMCboxDimensions{size=.32cm,down=0pt,rule=.2pt}\AMCbin@sequence={}\AMCbin@number=#2\relax%
\AMCbin@ndigits=\z@%
\loop%
\ifnum\AMCbin@number>\z@%
\advance\AMCbin@ndigits\@ne%
\ifodd\AMCbin@number\AMCbin@sequence=\expandafter{\expandafter\AMCbin@one\the\AMCbin@sequence}%
\else\AMCbin@sequence=\expandafter{\expandafter\AMCbin@zero\the\AMCbin@sequence}\fi%
\divide\AMCbin@number\tw@%
\repeat%
\loop\relax%
\ifnum\AMCbin@ndigits<#1\advance\AMCbin@ndigits\@ne%
\AMCbin@sequence=\expandafter{\expandafter\AMCbin@zero\the\AMCbin@sequence}\repeat%
\the\AMCbin@sequence%
}}
%    \end{macrocode}\end{macro}
%
% \subsection{Handling groups of questions}
%
% The package allows to handle groups of questions, so as to be able
% to shuffle them before printing them to the sheets.
%
% \begin{macro}{\nouveaugroupe}\begin{macro}{\element}
%     Command |\nouveaugroupe|\marg{group-name}\marg{n} creates a new
%     (empty) group with name \meta{group-name} (argument \meta{n} is
%     present only for compatibility reasons and is ignored). Command
%     |\element|\marg{group-name}\marg{text} adds to group
%     \meta{group-name} a new element that contains
%     \meta{text}. \meta{text} can be a |question| environment, ore
%     two successive |question|s to be kept together, or
%     anything else. Calling command |\nouveaugroupe| is not
%     compulsory, as |\element| calls it if necessary.
%    \begin{macrocode}
\newcount\AMCtok@k
\newcount\AMCtok@max
\newcommand{\nouveaugroupe}[2]{%
  \expandafter\ifx\csname #1@k\endcsname\relax
    \expandafter\newcount\csname #1@k\endcsname%
    \csname #1@k\endcsname=\z@%
  \fi%
}
\newcommand\AMC@prepare@element[1]{%
  \nouveaugroupe{#1}{}%
  \global\advance\csname #1@k\endcsname\@ne\relax%
  \AMCtok@k=\csname #1@k\endcsname%
  \expandafter\ifx\csname #1@\romannumeral\AMCtok@k\endcsname\relax%
    \expandafter\newtoks\csname #1@\romannumeral\AMCtok@k\endcsname\fi%
}
\newcommand{\element}[2]{%
  \AMC@prepare@element{#1}%
  \csname #1@\romannumeral\AMCtok@k\endcsname={#2}%
}
%    \end{macrocode}\end{macro}\end{macro}
% \begin{macro}{\shufflegroup}\begin{macro}{\insertgroup}
%     Command |\shufflegroup|\marg{group-name} shuffles the elements
%     of group \meta{group-name}. It can be called at each student
%     sheet in order to get different student sheets and avoid
%     cheating.  Command |\insertgroup|\oarg{n}\marg{groupname}
%     inserts all the elements of group \meta{groupname}, or only the
%     first \meta{n} elements if \meta{n} is given.
%    \begin{macrocode}
\newcommand{\shufflegroup}[1]{%
  {\AMC@shuffletoks{\number\csname #1@k\endcsname}{#1@}}%
}
\newcount\AMCtok@ik
\newcommand{\insertgroup}[2][0]{%
  \AMCtok@max=#1\relax%
  \ifnum\the\AMCtok@max<1%
    \AMCtok@max=\csname #2@k\endcsname%
  \fi%
  \AMCtok@ik=\z@%
  {\loop%
    \advance\AMCtok@ik\@ne\relax%
    \the\csname #2@\romannumeral\AMCtok@ik\endcsname%
  \ifnum\AMCtok@ik<\AMCtok@max\repeat}%
}
%    \end{macrocode}\end{macro}\end{macro}
% 
% \begin{macro}{\cleargroup}\begin{macro}{\copygroup}
%     The commands |\cleargroup| and |\copygroup| can also be used to
%     make more complex questions combinations in the exams, allowing
%     for example to ask the package to shuffle 3 questions taken at
%     random from group |groupa| and 5 questions taken at random from
%     group |groupb|.
%
%     |\cleargroup|\marg{group} clears the group \meta{group},
%     ereasing all of its elements.
%
%     |\copygroup|\oarg{n}\marg{from}\marg{to} copies \meta{n}
%     elements from group \meta{from} to group \meta{to}. If optional
%     parameter \meta{n} is not given, all the questions from group
%     \meta{from} are copied.
%
%     See section~\ref{d:groups} for an illustration for these commands.
%
%    \begin{macrocode}
\newcommand{\cleargroup}[1]{%
  \nouveaugroupe{#1}{}%
  \csname #1@k\endcsname=\z@%
}
\newcommand{\copygroup}[3][0]{%
  \AMCtok@max=#1\relax%
  \ifnum\the\AMCtok@max<1%
    \AMCtok@max=\csname #2@k\endcsname%
  \fi%
  \AMCtok@ik=\z@%
  {\loop%
    \advance\AMCtok@ik\@ne\relax%
    \AMC@prepare@element{#3}%    
    \global\csname #3@\romannumeral\AMCtok@k\endcsname=\csname #2@\romannumeral\AMCtok@ik\endcsname%
  \ifnum\AMCtok@ik<\AMCtok@max\repeat}%
}
%    \end{macrocode}
% \end{macro}
% \end{macro}
% 
% \subsection{Questions}
%
% To manage multiple choice questions, first set some counters and
% token registers to handle answers. Token registers |\reponse@i|,
% |\reponse@ii| and so on will be used for answers -- we restrict the
% number of answers of a single questions to
% |\AMCload@counter|${}=199$.
%    \begin{macrocode}
\newcount\AMCrep@count
\AMCload@counter=199
\@whilenum\AMCload@counter>0\do{%
  \expandafter\newtoks\csname reponse@\romannumeral\AMCload@counter\endcsname%
  \advance\AMCload@counter\m@ne%
}
%    \end{macrocode}
% \begin{macro}{\AMCload@reponse}\begin{macro}{\AMCrien@deux}
%     Command |\AMCload@reponse|\marg{n}\marg{text} will be used to
%     add answer number \meta{n} with text \meta{text} (\meta{text}
%     will include the box to be ticked and all the layout commands)
%     to the set of answers (in a token register |\reponse@|\emph{xxx}
%     -- counter |\AMCload@counter| keeps track of the number of
%     answers), in order to shuffle them when all answers will be
%     loaded.
%
%     When answers are not to be shuffled, command
%     |\AMCrien@deux|\marg{n}\marg{text} will be used instead, only
%     printing \meta{text}.
%    \begin{macrocode}
\newcommand\AMCload@reponse[2]{%
  \advance\AMCload@counter\@ne\relax%
  \csname reponse@\romannumeral\AMCload@counter\endcsname%
  =\expandafter{\expandafter\AMCrep@count\expandafter=#2 #1}%
}
\newcommand\AMCrien@deux[2]{#1}
%    \end{macrocode}\end{macro}\end{macro}
% \begin{macro}{\shuffle@it}\begin{macro}{\AMCdump@reponses}
%     After loading all answers, commands |\shuffle@it| will be used to
%     shuffle them, and |\AMCdump@reponses| to print them.
%    \begin{macrocode}
\def\shuffle@it{\AMC@shuffletoks{\number\AMCload@counter}{reponse@}}
\newcount\AMCnum@questions
\newcommand\AMCdump@reponses{%
  \global\AMCnum@questions=\AMCload@counter%
  \@whilenum\AMCload@counter>0\do{%
    \the\csname reponse@\romannumeral\AMCload@counter\endcsname%
    \advance\AMCload@counter\m@ne}}
%    \end{macrocode}\end{macro}\end{macro}
%
% \subsubsection{Managing answers}
% 
% \begin{macro}{\lastchoices}\begin{macro}{\AMCrep@init}\begin{macro}{\AMC@fin@rep}
%       Command |\AMCrep@init|\marg{mode} is called for each question
%       before reading answers. \meta{mode} is |r| for suffled
%       answers, and |o| if answers are not to be shuffled. It sets
%       the number of answers counter to zero, and calls |\AMCrep@o|
%       or |\AMCrep@r| depending on \meta{mode}. These commands sets
%       |\AMCload@@reponse| and |\AMCrep@fini| that will be called for
%       each answer and after the last answer respectively, depending
%       on \meta{mode}:
%       \begin{itemize}
%       \item If \meta{mode}=|r|, |\AMCload@@reponse| is
%         |\AMCload@reponse| (loads answer to token register) and
%         |\AMCrep@fini| calls |\shuffle@it| and |\AMCdump@reponses|;
%       \item If \meta{mode}=|o|, |\AMCload@@reponse| is
%         |\AMCrien@deux| (prints answer directly) and |\AMCrep@fini|
%         does nothing.
%       \end{itemize}
%
%       Command |\lastchoices| is called before giving answers that
%       are to be printed at the end (even when shuffling answers). It
%       closes the answers list calling |\AMCrep@fini| and opens
%       another one in ordered mode. Note that it also saves the value
%       of |\AMCrep@count|, which is the number of the current answer
%       among all answers given in the subject source for the current
%       question.
%
%       Command |\AMC@fin@rep| is to be called after the last answer:
%       it adds a ``\makeatletter\AMC@loc@none\makeatother'' answer if
%       necessary (package option |completemulti|) with answer number
%       zero, and calls |\AMCrep@fini|.
%    \begin{macrocode}
\newcommand\AMCrep@init[1]{%
  \ifAMC@ordre\AMCrep@o\else%
    \csname AMCrep@#1\endcsname\fi\AMCload@counter=\z@}
\newcommand\AMCrep@o{%
  \def\AMCload@@reponse{\AMCrien@deux}\def\AMCrep@fini{}}
\newcommand\AMCrep@r{%
  \def\AMCload@@reponse{\AMCload@reponse}%
  \def\AMCrep@fini{\shuffle@it\AMCdump@reponses}}
\newcount\AMCrep@@count
\newcommand\lastchoices{%
  \AMCrep@@count=\AMCrep@count%
  \AMCrep@fini\AMCrep@init{o}%
  \AMCrep@count=\AMCrep@@count}
\newcommand\@aucune{\emph{\AMC@loc@none}}
\newcommand\AMC@fin@rep{%
  \ifAMCcomplete@multi\ifAMCtype@multi%
    \lastchoices\AMCrep@count=-1%
    \ifAMCune@bonne\wrongchoice{\@aucune}\else%
      \ifAMC@postcorrect\wrongchoice{\@aucune}\else\correctchoice{\@aucune}\fi%
    \fi\fi\fi\AMCrep@fini}
%    \end{macrocode}\end{macro}\end{macro}\end{macro}
%
% \subsubsection{Separate answer sheet}
%
% This package needs some memory to print questions/answers boxes
% again on a separate answer sheet.
% 
% \begin{macro}{\AMCformQuestion}\begin{macro}{\AMCformAnswer}
%     First define commands that will announce questions and answers
%     on the separate answer sheet (these commands can be modified by
%     the user): |\AMCformQuestion|\marg{n} is responsible for
%     announcing question number \meta{n}, and
%     |\AMCformAnswer|\marg{box} is responsible for printing the box
%     to be ticked, given as argument \meta{box}.
%
%     Commands |\AMCformQuestionA| and |\AMCformAnswerA| set up
%     counter |\AMC@ncase| value before calling their counterparts.
%
%    \begin{macrocode}
\def\AMCmem@ireData{}
\def\AMCformQuestion#1{\vspace{\AMCformVSpace}\par{\AMC@loc@qf{#1}}}
\def\AMCformQuestionA#1{\setcounter{AMC@ncase}{0}\AMCformQuestion{#1}}
\def\AMCformAnswer#1{\hspace{\AMCformHSpace} #1}
\def\AMCformAnswerA#1{\addtocounter{AMC@ncase}{1}\AMCformAnswer{#1}}
%    \end{macrocode}\end{macro}\end{macro}\
%
% \begin{macro}{\AMCmem@ireAJ}\begin{macro}{\AMCformBegin}\begin{macro}{\AMCform}\begin{macro}{\AMCformS}
%       These are commands to manage memory for separate answer
%       sheet. |\AMCmem@ireAJ|\marg{code} adds \meta{code} to this
%       memory. |\AMCmem@ireAJRep|\marg{code} adds to memory answer
%       code \meta{code}, and |\AMCmem@ireQ|\marg{n} adds to memory
%       question code to announce question number \meta{n}.
%
%       The command |\AMCformBegin| defines the beginning of the
%       separate answer sheet for the current student sheet, and
%       |\AMCform| prints the whole memory: questions and answers
%       boxes.
%
%       |\AMCformS| is a |\AMCform| variant that does not clear the
%       list of answer boxes. It can be used to make the same exact
%       subject for all students, displaying the questions before
%       (outside) |onecopy|, so that |onecopy| contains only the
%       answer sheet.
%
%    \begin{macrocode}
\newcommand\AMCmem@ireAJ[1]{%
  \ifAMC@ensemble\ifAMC@zoneformulaire\else%
    \begingroup\AMCformulaire@dedanstrue%
      \let\protect\@unexpandable@protect%
      \global\edef\AMCmem@ireData{\AMCmem@ireData #1}%
    \endgroup\fi\fi}
\newcommand\AMCmem@ireAJRep[1]{%
  \addtocounter{AMC@ncase}{1}\AMCmem@ireAJ{\protect\AMCformAnswerA{#1}}}
\newcommand\AMCmem@ireQ[1]{\AMCmem@ireAJ{\protect\AMCformQuestionA{#1}}}
\def\AMCformBegin{\AMC@zoneformulairetrue}
\newcommand\AMCform{%
  \ifAMC@ensemble\AMCformulaire@dedanstrue\AMCmem@ireData%
  \global\def\AMCmem@ireData{}\fi}
\newcommand\AMCformS{%
  \ifAMC@ensemble\AMCformulaire@dedanstrue%
  \AMC@amclog{AUTOQCM[BR=0]^^J}\AMCmem@ireData%
  \fi}
%    \end{macrocode}\end{macro}\end{macro}\end{macro}\end{macro}
%
% \subsubsection{Formatting answers}
%
% \begin{environment}{choices}\begin{environment}{choiceshoriz}\begin{environment}{choicescustom}\begin{macro}{\AMCBoxedAnswers}
%       Answers have to be included in an environment |choices|
%       (standard), |choiceshoriz| (answers on one line) or
%       |choicescustom| (user defined) depending on the desired
%       formatting.
%
%       Use |\AMCBoxedAnswers| to request all answers to be included
%       in \LaTeX{} boxes; this can be useful for example when using
%       multicolumn answers formatting.
%    \begin{macrocode}
\def\AMCBoxedAnswers{\AMC@rbloctrue}
\newenvironment{choices}[1][r]{%
  \AMCrep@count=\z@\def\une@rep{\AMCrep@itemize}%
  \ifAMC@rbloc\def\une@rep{\AMCrep@bloc}%
  \else\begin{itemize}\setlength{\itemsep}{\AMCinterIrep}\fi%
    \AMCrep@init{#1}}%
  {\AMC@fin@rep\ifAMC@rbloc\else\end{itemize}\fi}
\newenvironment{choiceshoriz}[1][r]{%
  \AMCrep@count=\z@\def\une@rep{\AMCrep@ligne}\AMCrep@init{#1}%
  \par\begin{center}}%
  {\AMC@fin@rep\end{center}}
\newenvironment{choicescustom}[1][r]{%
  \AMCrep@count=\z@\def\une@rep{\AMCrep@perso}\AMCrep@init{#1}%
  \AMCbeginAnswer\ignorespaces}%
  {\AMC@fin@rep\AMCendAnswer}
%    \end{macrocode}\end{macro}\end{environment}\end{environment}\end{environment}
%
% \begin{macro}{\AMCrep@bloc}\begin{macro}{\AMCrep@itemize}
% \begin{macro}{\AMCrep@ligne}\begin{macro}{\AMCrep@perso}
%     For each of these styles, a corresponding
%     |\AMCrep@|$xxx$\marg{box}\marg{text} is defined, which will
%     format the answer with a box given in \meta{box} and text
%     \meta{text}. |\AMCrep@bloc| is also defined and used in standard
%     formatting when the user wants to put answers inside a \LaTeX{}
%     box.
%    \begin{macrocode}
\newcommand\AMCrep@bloc[2]{\AMCmem@ireAJRep{#1}%
  \par\noindent\begin{minipage}{\linewidth}%
    \begin{itemize}\item[#1] #2\end{itemize}\end{minipage}%
  \vspace{\AMCinterBrep}}
\newcommand\AMCrep@itemize[2]{\AMCmem@ireAJRep{#1}\item[#1] #2}
\newcommand\AMCrep@ligne[2]{\AMCmem@ireAJRep{#1}%
  \mbox{#1\hspace*{1em}#2}\hspace{3em plus 4em}}
\newcommand\AMCrep@perso[2]{\AMCmem@ireAJRep{#1}\AMCanswer{#1}{#2}}
%    \end{macrocode}\end{macro}\end{macro}\end{macro}\end{macro}
%
% \begin{macro}{\AMCbeginAnswer}\begin{macro}{\AMCendAnswer}\begin{macro}{\AMCanswer}
%       The |custom| style will use user-defined commands to format
%       answers: |\AMCbeginAnswer| is called once before answers,
%       |\AMCanswer|\marg{box}\marg{text} is called for each answer
%       (\meta{box} beeing the box to be ticked and \meta{text} the
%       text associated with the proposed answer), and |\AMCendAnswer|
%       is called after all answers.
%    \begin{macrocode}
\def\AMCbeginAnswer{}
\def\AMCanswer#1#2{#1 #2}
\def\AMCendAnswer{}
%    \end{macrocode}\end{macro}\end{macro}\end{macro}
%
% \begin{macro}{\correctchoice}\begin{macro}{\wrongchoice}
%     The commands |\correctchoice| and |\wrongchoice| are used inside
%     |choices|-like environments to give the proposed answers and
%     specify if they are to be tocked by the students or not.
%
%    \begin{macrocode}
\newcommand{\correctchoice}[2][]{\global\advance\AMCrep@count\@ne\relax%
  \ifAMC@calibration\AMC@amclog{AUTOQCM[REP=\the\AMCrep@count:B]^^J}\fi%
  \global\AMCune@bonnetrue%
  \AMCload@@reponse{\une@rep{\ifAMC@correc\AMC@marque{#1}{1}%
      \else\AMC@marque{#1}{}\fi}{#2}}{\the\AMCrep@count}\ignorespaces}
\newcommand{\wrongchoice}[2][]{\global\advance\AMCrep@count\@ne\relax%
  \ifAMC@calibration\AMC@amclog{AUTOQCM[REP=\the\AMCrep@count:M]^^J}\fi%
  \AMCload@@reponse{\une@rep{\AMC@marque{#1}{}}{#2}}{\the\AMCrep@count}%
  \ignorespaces}
%    \end{macrocode}\end{macro}\end{macro}
%
% \subsubsection{Formatting questions}
%
% \begin{macro}{\AMCquestionaff}\begin{macro}{\AMC@qaff}
%   The counter |\AMCquestionaff| keeps track of the current question
%   number. It can be redefined by the user, for example to print
%   several questions without a number, and then print questions with a
%   number starting at one.
%   
%   |\AMC@qaff| will increase this counter and format the question
%   number out.
%    \begin{macrocode}
\newcounter{AMCquestionaff}
\newcommand{\AMCnumero}[1]{\setcounter{AMCquestionaff}{#1}\addtocounter{AMCquestionaff}{-1}}
\newcommand\AMC@qaff{\addtocounter{AMCquestionaff}{1}\arabic{AMCquestionaff}}
%    \end{macrocode}\end{macro}\end{macro}
%
% \begin{macro}{\AMCbeginQuestion}\begin{macro}{\multiSymbole}
%   The command |\AMCbeginQuestion|\marg{n}\marg{sign} will format the
%   question header, where \meta{n} is the question number and
%   \meta{sign} beeing |\multiSymbole| in case of a multiple question,
%   and empty in case of a simple one. |\AMCbeginQuestion| and
%   |\multiSymbole| can be user-redifined.
%    \begin{macrocode}
\def\AMCbeginQuestion#1#2{\par\noindent\AMC@loc@q{#1}{#2}\hspace*{1em}}
\def\multiSymbole{$\clubsuit$}
%    \end{macrocode}\end{macro}\end{macro}
%
% \begin{environment}{question}\begin{environment}{questionmult}\begin{environment}{questionouverte}\begin{macro}{\ouverte@vs}
%         Environment |{question}|\marg{key} encloses a simple question
%         (with one and only one correct choice) with associated
%         unique key \meta{key} and the proposed answers.
%
%         Environment |{questionmult}|\marg{key} is the same for
%         multiple questions (with none, one or several correct
%         choices).
%
%         Environment |{questionmultx}|\marg{key} is the same as
%         |questionmult|, but with no use of |\multiSymbole|.
%
%         Environment |{questionouverte}|\oarg{width} is used for open
%         questions (that won't be marked automatically!), with width
%         given as an optional argument (defaults to 3\,cm).
%    \begin{macrocode}
\ifx\question\undefined\else\let\question\undefined\fi
\def\AMCnobloc{\AMC@qblocfalse}
\def\AMCbloc{\AMC@qbloctrue}
\newenvironment{question}[2][]{%
  \global\def\AMCid@name{#2}\AMC@affecte{#2}{\AMCid@quest}%
  \ifAMC@calibration\AMCmessage{Q=\the\AMCid@quest}\fi%
  \AMCtype@multifalse\ifAMC@qbloc\noindent\begin{minipage}{\linewidth}\fi%
  \ifAMC@affichekeys\index{\texttt{#2}}\fi%
  \AMCbeginQuestion{\ifAMC@affichekeys[\texttt{#2}]\else\AMC@qaff\fi}{#1}%
  \AMCformulaire@dedansfalse\setcounter{AMC@ncase}{0}%
  \AMCmem@ireQ{\arabic{AMCquestionaff}}}%
{\ifAMC@qbloc\end{minipage}\vspace{3ex}\fi\AMCmessage{FQ}}
\newenvironment{questionmult}[1]{%
  \AMCune@bonnefalse\begin{question}[{{\multiSymbole}}]{#1}%
  \AMCtype@multitrue\ifAMC@calibration%
  \AMC@amclog{AUTOQCM[MULT]^^J}\fi}%
{\end{question}}
\newenvironment{questionmultx}[1]{%
  \begingroup\def\multiSymbole{}\begin{questionmult}{#1}}%
{\end{questionmult}\endgroup}
\newdimen\ouverte@vs
\newenvironment{questionouverte}[1][3cm]{%
  \AMCtype@multifalse\ouverte@vs=#1%
  \ifAMC@qbloc\noindent\begin{minipage}{\linewidth}\fi%
  \AMCbeginQuestion{\AMC@qaff}{}}%
{\vspace*{\ouverte@vs}\ifAMC@qbloc\end{minipage}\vspace{3ex}\fi}
%    \end{macrocode}\end{macro}\end{environment}\end{environment}\end{environment}
%
% \subsection{Scoring}
%
% \begin{macro}{\scoring}\begin{macro}{\scoringDefaultS}\begin{macro}{\scoringDefaultM}\begin{macro}{\QuestionIndicative}
%         Scoring strategies are simply transmitted to the |.amc| file
%         for later analysis.
%         
%         |\scoring|\marg{scrore} details the scoring strategy for
%         current question or current answer,
%         |\scoringDefaultS|\marg{score} and
%         |\scoringDefaultM|\marg{score} gives default scoring
%         strategy for simple and multiple questions, and
%         |\QuestionIndicative| tells that the current question is not
%         no be taken into account in the global mark.
%    \begin{macrocode}
\def\scoring#1{\ifAMC@calibration\AMC@amclog{AUTOQCM[B=#1]^^J}\fi}
\def\scoringDefaultS#1{\ifAMC@calibration\AMC@amclog{AUTOQCM[BDS=#1]^^J}\fi}
\def\scoringDefaultM#1{\ifAMC@calibration\AMC@amclog{AUTOQCM[BDM=#1]^^J}\fi}
\def\QuestionIndicative{\ifAMC@calibration\AMC@amclog{AUTOQCM[INDIC]^^J}\fi}
%    \end{macrocode}\end{macro}\end{macro}\end{macro}\end{macro}
%
% \subsection{Numerical data}
%
% \subsubsection{Codes}
%
% \begin{macro}{\AMCcode}\begin{macro}{\AMCcodeH}
% \noindent\begin{minipage}[t]{.45\linewidth}
% Students can code some numerical information (such as student
% number) through special questions, which can be formatted easily
% with the command |\AMCcode|\marg{key}\marg{ndigits}, where
% \meta{key} is a key prefix and \meta{ndigits} is the number of
% required digits. The digits entered by the student will be available
% through the questions \meta{key}|.1|$,\ldots,$\meta{key}|.|\meta{ndigits}.
%
% As an example, |\AMCcode{code}{6}| produces the opposite boxes (two
% results are show here: without or with |separateanswersheet|
% option), and trace positions of all the boxes in the |.xy| file with
% the |code| identifier: the first digit is represented by question
% with key |code.6|, the second by question with key |code.5|, and so on.
%
% Positions of the boxes are logged in the |.xy| file, as shown in
% section~\ref{a:code} for the first set of boxes (without
% |separateanswersheet|, with digits outside boxes).
% \end{minipage}\hfill
% \begin{minipage}[t]{.45\linewidth}
%   \begin{amcxyfile}{automultiplechoice.xy3}\AMCcode{code}{6}\end{amcxyfile}
%
% \vspace{4mm}
%   \makeatletter\AMC@ensembletrue\makeatother
%   \AMCcode{code}{6}
% \end{minipage}
% \vspace{4mm}
%
% \noindent\begin{minipage}[t]{.35\linewidth}
%   The ``horizontal'' version |\AMCcodeH| can also be considered,
%   specially with a small number of digits. See opposite for the
%   result of |\AMCcodeH{code}{3}|.
% \end{minipage}\hfill
% \begin{minipage}[t]{.6\linewidth}
%   \AMCcodeH{code}{3}
% \end{minipage}
% 
%    \begin{macrocode}
\newcount\AMC@chiffres
\newdimen\AMCcodeHspace\AMCcodeHspace=.5em
\newdimen\AMCcodeVspace\AMCcodeVspace=.5em
\newdimen\AMCcodeBoxSep\AMCcodeBoxSep=.1em
\newcommand{\AMCcode}[2]{%
{\def\AMCbeginQuestion##1##2{}%
 \setlength{\parindent}{0pt}%
 \def\AMCbeginAnswer{\hspace{0pt}%
    \vbox\bgroup}%
 \def\AMCendAnswer{\vspace{-\AMCcodeVspace}\egroup%
    \hspace{\AMCcodeHspace}}%
 \def\AMCanswer##1##2{\hbox{##1\ifAMC@ensemble\else%
   \ifAMC@inside@box\else{\hspace{\AMCcodeBoxSep}\textbf{##2}}\fi\fi\hspace*{\fill}}%
   \vspace{\AMCcodeVspace}}%
 \AMCnobloc%
 \AMC@chiffres=#2\loop%
 \begin{question}{#1.\the\AMC@chiffres}\QuestionIndicative%
   \begin{choicescustom}[o]\scoring{auto=0}%
     \wrongchoice[0]{0}%
     \wrongchoice[1]{1}%
     \wrongchoice[2]{2}%
     \wrongchoice[3]{3}%
     \wrongchoice[4]{4}%
     \wrongchoice[5]{5}%
     \wrongchoice[6]{6}%
     \wrongchoice[7]{7}%
     \wrongchoice[8]{8}%
     \wrongchoice[9]{9}%
   \end{choicescustom}%
 \end{question}%
 \advance\AMC@chiffres\m@ne\ifnum\AMC@chiffres>0\repeat%
 \hspace{-\AMCcodeHspace}%
}}
\newcommand{\AMCcodeH}[2]{%
{\def\AMCbeginQuestion##1##2{}%
 \setlength{\parindent}{0pt}%
 \def\AMCbeginAnswer{\hbox\bgroup}%
 \def\AMCendAnswer{\egroup\vspace{\AMCcodeVspace}\par}%
 \def\AMCanswer##1##2{\hbox{##1\ifAMC@ensemble\else%
   \ifAMC@inside@box\else\hspace{\AMCcodeBoxSep}\textbf{##2}\fi\fi}%
   \hspace{\AMCcodeHspace}}%
 \AMCnobloc%
 \AMC@chiffres=#2\loop%
 \begin{question}{#1.\the\AMC@chiffres}\QuestionIndicative%
   \begin{choicescustom}[o]\scoring{auto=0}%
     \wrongchoice[0]{0}%
     \wrongchoice[1]{1}%
     \wrongchoice[2]{2}%
     \wrongchoice[3]{3}%
     \wrongchoice[4]{4}%
     \wrongchoice[5]{5}%
     \wrongchoice[6]{6}%
     \wrongchoice[7]{7}%
     \wrongchoice[8]{8}%
     \wrongchoice[9]{9}%
   \end{choicescustom}%
 \end{question}%
 \advance\AMC@chiffres\m@ne\ifnum\AMC@chiffres>0\repeat%
}}
%    \end{macrocode}\end{macro}\end{macro}
%
% \subsubsection{Numerical questions}
%
% \begin{macro}{\AMCnumericChoices}
%   The command |\AMCnumericChoices|\marg{correct}\marg{options} can be
%   used as a replacement for the |choices| environment when the
%   questions asks for a numeric value to code on the answer sheet.
%
%   As an example, 
%\iffalse
%<*doc>
%\fi
\begin{verbatim}
\begin{question}{product}
  What is the value of $7\times 5$?
  \AMCnumericChoices{35}{digits=2,sign=false}
\end{question}
\end{verbatim}
%\iffalse
%</doc>
%\fi
% produces (in correction mode):
%
% \begin{center}
% {\makeatletter\AMC@correctrue\makeatother
% \fbox{
% \begin{minipage}{.7\linewidth}
%   \begin{question}{product}
%     What is the value of $7\times 5$?
%     \AMCnumericChoices{35}{digits=2,sign=false}
%   \end{question}
% \end{minipage}}}
% \end{center}
%
% \noindent and these boxes are only shown on the separate answer sheet if the
% |separateanswersheet| option is used.
%
% This command uses the
% |\AMCformatChoices|\marg{showcommand}\marg{hidecommand}\marg{qname}\marg{qid}
% command, that calls either \meta{hidecommand} if the
% |separateanswersheet| option is used and if we are currently in the
% question part (not in the answer sheet), or \meta{showcommand} when
% all the boxes are to be produced.
% 
%    \begin{macrocode}
\newcommand\AMCformatChoices[4]{%
  \global\AMCrep@count=\z@%
  \AMCmem@ireAJ{\global\AMCrep@count=\z@%
    \protect #1{#3}{#4}{\AMCid@name}{\the\AMCid@quest}}%
  \ifAMC@ensemble%
    #2{#3}{#4}{\AMCid@name}{\the\AMCid@quest}%
    \AMC@amclog{AUTOQCM[QPART]^^J}%
  \else%
    #1{#3}{#4}{\AMCid@name}{\the\AMCid@quest}%
  \fi%
}
%    \end{macrocode}
%
% The |\AMCnumeric@char|\marg{inside}\marg{correct} draw a box with
% content \meta{inside} (only if needed), where \meta{correct} is |1|
% if the corresponding choice is correct and empty if not.
%
%    \begin{macrocode}
\newcommand{\AMCnumeric@char}[2]{%
  \global\advance\AMCrep@count\@ne\relax%
  \AMC@amclog{AUTOQCM[REP=\the\AMCrep@count:\ifx#2\@empty\@empty M\else B\fi]^^J}%
  \ifAMC@correc\AMC@marque{#1}{#2}\else\AMC@marque{#1}{}\fi%
  \ifAMC@inside@box\else\ifAMC@ensemble\else%
    \hspace{\AMCcodeBoxSep}\textbf{#1}\fi\fi%
}
%    \end{macrocode}
% 
% The command |\AMCnumeric@digit|\marg{correct}\marg{maxdigit} draws a
% box for current digit value |\AMC@chiffres|, if \meta{correct} is
% the correct digit value, and \meta{maxdigit} is the maximal digit
% value. The command |\AMCsignV|\marg{valuecount} draws two boxes for
% the students to code the sign of the counter \meta{valuecount}
% (which will be set to the absolute value). The command
% |\AMCnumericH|\marg{varname}\marg{correct}\marg{maxdigit} draws a
% serie of boxes for all possible values of a digit (from 0 to
% \meta{maxdigit}), where the correct value is \meta{correct},
% transmitting scoring data to AMC so that the vaiable \meta{varname}
% will be set to the value chosen by the student.
%
%    \begin{macrocode}
\newdimen\AMCnumeric@Hspace\AMCnumeric@Hspace=.5em
\newdimen\AMCnumeric@Vspace\AMCnumeric@Vspace=1ex
\newcommand{\AMCnumeric@digit}[2]{%
  \ifnum\AMC@chiffres=#1%
    \AMCnumeric@char{\the\AMC@chiffres}{1}%
  \else%
    \AMCnumeric@char{\the\AMC@chiffres}{}%
  \fi%
}
\newcommand{\AMCsignV}[1]{%
  \ifnum#1<\z@%
    \hbox{\AMCnumeric@char{$+$}{}}\vspace{\AMCnumeric@Vspace}
    \AMC@amclog{AUTOQCM[B=set.intS=1]^^J}%
    \hbox{\AMCnumeric@char{$-$}{1}}
    \AMC@amclog{AUTOQCM[B=set.intS=-1]^^J}%
    \global\multiply#1\m@ne%
  \else%
    \hbox{\AMCnumeric@char{$+$}{1}}\vspace{\AMCnumeric@Vspace}
    \AMC@amclog{AUTOQCM[B=set.intS=1]^^J}%
    \hbox{\AMCnumeric@char{$-$}{}}
    \AMC@amclog{AUTOQCM[B=set.intS=-1]^^J}%
  \fi%
}
\newcommand{\AMCnumericH}[3]{%
 \AMC@chiffres=0\loop%
   \AMCnumeric@digit{#2}{#3}%
   \AMC@amclog{AUTOQCM[B=set.#1=\the\AMC@chiffres]^^J}%
 \advance\AMC@chiffres\@ne%
 \ifnum\AMC@chiffres<#3\relax\hspace{\AMCnumeric@Hspace}\repeat%
}
\newcommand{\AMCnumericV}[3]{%
 \AMC@chiffres=0\loop%
   \vbox{\hbox{\AMCnumeric@digit{#2}{#3}}}%
   \AMC@amclog{AUTOQCM[B=set.#1=\the\AMC@chiffres]^^J}%
 \advance\AMC@chiffres\@ne%
 \ifnum\AMC@chiffres<#3\relax\vspace{\AMCnumeric@Vspace}\repeat%
}
\newcommand{\AMCnumericVR}[3]{%
 \AMC@chiffres=#3\advance\AMC@chiffres\m@ne\loop%
   \vbox{\hbox{\AMCnumeric@digit{#2}{#3}}}%
   \AMC@amclog{AUTOQCM[B=set.#1=\the\AMC@chiffres]^^J}%
 \ifnum\AMC@chiffres>0%
   \advance\AMC@chiffres\m@ne%
 \vspace{\AMCnumeric@Vspace}\repeat%
}
%    \end{macrocode}
%
% Some computation commands are now
% defined. |\AMC@calcmodulo|\marg{integer}\marg{counter} sets the
% counter \meta{counter} value to the last digit of \meta{integer}
% (using base
% |\AMC@numeric@base|). |\AMC@calcdigit|\marg{integer}\marg{digit}\marg{counter}
% sets the value of the counter \meta{counter} to the digit number
% \meta{digit} of \meta{integer} (digit number 0 of 567 is 7, number 1
% is 6...)
%
%    \begin{macrocode}
\newcount\AMC@integer@modulo
\newcommand\AMC@calcmodulo[2]{%
  \AMC@integer@modulo=#1\divide\AMC@integer@modulo\AMC@numeric@base%
  \multiply\AMC@integer@modulo\AMC@numeric@base%
  \multiply\AMC@integer@modulo\m@ne\advance\AMC@integer@modulo by #1\relax%
  #2=\AMC@integer@modulo%
}
\newcount\AMC@numeric@integer
\newcount\AMC@numeric@calcdigit
\newcommand\AMC@calcdigit[3]{%
  \AMC@numeric@integer=#1%
  \AMC@numeric@calcdigit=#2%
  \ifnum#2>\z@\loop{%
    \global\divide\AMC@numeric@integer\AMC@numeric@base}%
    \advance\AMC@numeric@calcdigit\m@ne%
  \ifnum\AMC@numeric@calcdigit>\z@\repeat%
  \fi%
  \AMC@calcmodulo{\the\AMC@numeric@integer}{#3}%
}
%    \end{macrocode}
%
% The command
% |\AMCnumericShow|\marg{value}\marg{opts}\marg{qname}\marg{qid} is
% called to draw all necessary boxes to code a numerical value
% \meta{value} with options given as a comma separated list
% \meta{opts}. |\AMCnumericOpts|\marg{opts} can be used to set some
% default values for these options.
%
% Begin with the available options:
%    \begin{macrocode}
\def\AMCdecimalPoint{\raisebox{1ex}{\bf .}}
\def\AMCntextSign{}
\def\AMCntextGoto{}
\def\AMCntextVHead#1{\emph{b#1}}
\def\AMCncol@Border{lightgray}
\def\AMCncol@Background{white}
\def\AMCncol@BorderWidth{1mm}
\newcount\AMC@numeric@digits
\newcount\AMC@numeric@decd
\newcount\AMC@numeric@value
\newcount\AMC@numeric@x
\newcount\AMC@numeric@base
\define@key{AMCNumeric}{Tsign}{\def\AMCntextSign{#1}}
\define@key{AMCNumeric}{Tpoint}{\def\AMCdecimalPoint{#1}}
\define@key{AMCNumeric}{vspace}{\AMCnumeric@Vspace=#1}
\define@key{AMCNumeric}{hspace}{\AMCnumeric@Hspace=#1}
\define@key{AMCNumeric}{bordercol}{\def\AMCncol@Border{#1}}
\define@key{AMCNumeric}{borderwidth}{\def\AMCncol@BorderWidth{#1}}
\define@key{AMCNumeric}{backgroundcol}{\def\AMCncol@Background{#1}}
\define@key{AMCNumeric}{digits}[3]{\AMC@numeric@digits=#1}
\define@key{AMCNumeric}{decimals}[0]{\AMC@numeric@decd=#1}
\define@key{AMCNumeric}{base}[10]{\AMC@numeric@base=#1}
\define@boolkey{AMCNumeric}{sign}[true]{}
\define@boolkey{AMCNumeric}{strict}[false]{}
\define@boolkey{AMCNumeric}{scoring}[true]{}
\define@boolkey{AMCNumeric}{vertical}[false]{}
\define@boolkey{AMCNumeric}{reverse}[true]{}
\define@boolkey{AMCNumeric}{vhead}[false]{}
\define@key{AMCNumeric}{scoreexact}[2]{\def\AMC@numeric@scoreexact{#1}}
\define@key{AMCNumeric}{scoreapprox}[1]{\def\AMC@numeric@scoreapprox{#1}}
\newcount\AMC@numeric@approx
\define@key{AMCNumeric}{approx}[0]{\AMC@numeric@approx=#1}
\setkeys{AMCNumeric}{digits,decimals,base,sign,strict,scoring,vertical,
                     reverse,vhead,scoreexact,scoreapprox,approx}
\newcommand\AMCnumericOpts[1]{\setkeys{AMCNumeric}{#1}}
%    \end{macrocode}
% Then the command |\AMCnumericShow| itself:
%    \begin{macrocode}
\newcommand\AMCnumericShow[4]{%
%    \end{macrocode}
% The first line allows to keep the question ID number and name
% accurate even in the separate answer sheet.
%    \begin{macrocode}
  \ifAMC@ensemble\def\AMCid@name{#3}\AMCid@quest=#4\fi%
%    \end{macrocode}
% We have to tell AMC that the scoring we will give concerns this question:
%    \begin{macrocode}
  \ifAMC@ensemble\ifAMCformulaire@dedans%
    \AMC@amclog{AUTOQCM[Q=\the\AMCid@quest]^^J}
  \fi\fi%
%    \end{macrocode}
% Then we parse the options from \meta{opts}:
%    \begin{macrocode}
  {\setkeys{AMCNumeric}{#2}%
%    \end{macrocode}
% When |decimal| is positive, convert the real correct value to integer.
%    \begin{macrocode}
  \ifnum\AMC@numeric@decd>\z@%
    \FPeval\AMC@numeric@eval{round(#1 * \the\AMC@numeric@base^\the\AMC@numeric@decd,0)}
    \AMC@numeric@value=\AMC@numeric@eval%
  \else%  
    \AMC@numeric@value=#1%
  \fi%
%    \end{macrocode}
% The question scoring is given to AMC (if requested by the
% |scoring=true| option). Note that the variable |intV| refers to the
% correct value, and |intX| to the value entered by the student.
%    \begin{macrocode}
  \ifKV@AMCNumeric@scoring%
    \AMC@amclog{AUTOQCM[B=formula=(intV==intX?%
      \AMC@numeric@scoreexact:%
      \ifnum\AMC@numeric@approx>\z@%
        (abs((intV)-(intX))<=\the\AMC@numeric@approx?\AMC@numeric@scoreapprox:0)%
      \else%
        0%
      \fi)]^^J}%
  \fi%
  \def\AMC@numeric@compute{}%
  \AMC@numeric@x=\AMC@numeric@digits\loop{%
    \ifKV@AMCNumeric@strict%
      \AMC@amclog{AUTOQCM[B=requires.int\@Alph\AMC@numeric@x=1]^^J}%
    \else%
      \AMC@amclog{AUTOQCM[B=default.int\@Alph\AMC@numeric@x=0]^^J}%
    \fi%
    \global\edef\AMC@numeric@compute{%
      \ifnum\AMC@numeric@x=\AMC@numeric@digits\else%
        (\AMC@numeric@compute)*\the\AMC@numeric@base+\fi%
      int\@Alph\AMC@numeric@x}%
  }\advance\AMC@numeric@x\m@ne\ifnum\AMC@numeric@x>0\repeat%
  \ifKV@AMCNumeric@sign%
    \ifKV@AMCNumeric@strict%
      \AMC@amclog{AUTOQCM[B=requires.intS=1]^^J}%
    \else%
      \AMC@amclog{AUTOQCM[B=default.intS=1]^^J}%
    \fi%
    \global\edef\AMC@numeric@compute{(\AMC@numeric@compute)*(intS)}%
  \fi%
  \AMC@amclog{AUTOQCM[B=set.intV=\the\AMC@numeric@value,%
    set.intX=\AMC@numeric@compute]^^J}%
%    \end{macrocode}
% Begin now with the frame around all the boxes:
%    \begin{macrocode}
  \vspace{1.5ex}\par{%
    \fboxrule=\AMCncol@BorderWidth%
    \fcolorbox{\AMCncol@Border}{\AMCncol@Background}{%
%    \end{macrocode}
% Place the boxes to choose the sign, if requested.
%    \begin{macrocode}
      \ifKV@AMCNumeric@sign%
        \vbox{%
          \ifx\AMCntextSign\@empty\@empty\else%
            \hbox{\AMCntextSign}\vspace{\AMCnumeric@Vspace}\fi%
          \AMCsignV{\AMC@numeric@value}}\hspace{.5em}%
        \vrule%
        \hspace{.5em}%
      \fi%
%    \end{macrocode}
% We shift |\AMC@numeric@digits| and |\AMC@numeric@decd| counters so
% that digit number 0 is the digit just before decimal point.
%    \begin{macrocode}
\advance\AMC@numeric@digits\m@ne%
\advance\AMC@numeric@decd\m@ne%
%    \end{macrocode}
% For vertical mode (boxes for a single digit are on a same row;
% usually used for binary numbers),
%    \begin{macrocode}
      \ifKV@AMCNumeric@vertical%
        \hbox{%
%    \end{macrocode}
% begin a loop over all digits,
%    \begin{macrocode}
          \loop{%
%    \end{macrocode}
% place the decimal point if necessary,
%    \begin{macrocode}
            \ifnum\AMC@numeric@digits=\AMC@numeric@decd\relax%
              \hbox{\AMCdecimalPoint}%
            \fi%
%    \end{macrocode}
% compute the digit value,
%    \begin{macrocode}
            \AMC@calcdigit{\the\AMC@numeric@value}%
              {\the\AMC@numeric@digits}{\AMC@numeric@x}%
%    \end{macrocode}
% draw the box for this digit,
%    \begin{macrocode}
            \hbox{\vbox{%
              \ifKV@AMCNumeric@vhead%
                \vbox{\hbox{\AMCntextVHead{\the\AMC@numeric@digits}}}%
                \vspace{\AMCnumeric@Vspace}%
              \fi%
              {\advance\AMC@numeric@digits\@ne%
               \ifKV@AMCNumeric@reverse%
                 \AMCnumericVR{int\@Alph\AMC@numeric@digits}%
                   {\the\AMC@numeric@x}{\AMC@numeric@base}%
               \else%
                 \AMCnumericV{int\@Alph\AMC@numeric@digits}%
                   {\the\AMC@numeric@x}{\AMC@numeric@base}%
               \fi}%
            }}%
%    \end{macrocode}
% and end the loop over digits, adding space if this is not the last one.
%    \begin{macrocode}
           }\ifnum\AMC@numeric@digits>\z@%
             \hspace{\AMCnumeric@Hspace}%
          \advance\AMC@numeric@digits\m@ne\repeat%
        }%
%    \end{macrocode}
% Now, do the same for horizontal mode.
%    \begin{macrocode}
       \else%
         \hbox{\vbox{%
           \loop{%
             \ifnum\AMC@numeric@digits=\AMC@numeric@decd\relax%
               \hbox{\AMCdecimalPoint}%
             \fi%
             \AMC@calcdigit{\the\AMC@numeric@value}%
               {\the\AMC@numeric@digits}{\AMC@numeric@x}%
             \hbox{%
               {\advance\AMC@numeric@digits\@ne%
                \AMCnumericH{int\@Alph\AMC@numeric@digits}%
                  {\the\AMC@numeric@x}{\AMC@numeric@base}%
             }}%
           }\ifnum\AMC@numeric@digits>\z@%
             \vspace{\AMCnumeric@Vspace}\par%
           \advance\AMC@numeric@digits\m@ne\repeat%
          }}%
      \fi%
%    \end{macrocode}
% Close the frame around all the boxes.
%    \begin{macrocode}
    }%
  }%
%    \end{macrocode}
% And tell AMC that we finished with this question:
%    \begin{macrocode}
  \ifAMC@ensemble\else\vspace{1.5ex}\par\fi%
  \ifAMC@ensemble\ifAMCformulaire@dedans%
    \AMC@amclog{AUTOQCM[FQ]^^J}%
  \fi\fi%
  }%
}
%    \end{macrocode}
%
% |\AMCnumericHide| is called when the boxes are not to be drawn (in
% the question sheets for separate answer sheet layout), and
% |\AMCnumericChoices|\marg{value}\marg{options} is the function to be
% used in the LaTeX source code of the exam.
%
%    \begin{macrocode}
\newcommand\AMCnumericHide[4]{%
  \setkeys{AMCNumeric}{#2}%
  \AMCntextGoto%
  \vspace{1.5ex}\par%
}
\def\AMCnumericChoices{\AMCformatChoices{\AMCnumericShow}{\AMCnumericHide}}
%    \end{macrocode}
% \end{macro}
%
% \subsubsection{Intervals}
%
% \begin{macro}{\AMCIntervals}
%   The command |\AMCIntervals|\marg{x}\marg{x0}\marg{x1}\marg{delta}
%   can be used to present answers as intervals $[x_i,x_i+\delta[$
%   covering $[\meta{x0},\meta{x1}[$, such that the only interval
%   containing \meta{x} is declared as |\correctchoice|, and the other
%   as |\wrongchoice|.
%
%   For this command to work, one has to load package \textsf{fp}.
%
%   As an example, 
%\iffalse
%<*doc>
%\fi
\begin{verbatim}
\begin{question}{quarter}
  In which interval falls $1/4$?
  \begin{multicols}{5}
    \begin{choices}[o]
      \AMCIntervals{0.25}{0}{1}{0.1}
    \end{choices}
  \end{multicols}
\end{question}
\end{verbatim}
%\iffalse
%</doc>
%\fi
%   produces (in correction mode):
%    \begin{question}{quarter}
%      \makeatletter\AMC@correctrue\makeatother
%      In which interval falls $1/4$?
%      \begin{multicols}{5}
%        \begin{choices}[o]
%          \AMCIntervals{0.25}{0}{1}{0.1}
%        \end{choices}
%      \end{multicols}
%    \end{question}
%
%    Note that the interval formatting can be changed redefining the
%    |\AMCintervalFormat| command, which is originally defined as
%    \begin{macrocode}
\def\AMCIntervalFormat#1#2{[#1,\,#2[}
%    \end{macrocode}
% to follow local conventions (writting $[a,b)$ instead of $[a,b[$ is
% for example a common usage).
%    \begin{macrocode}
\def\AMC@intervx#1#2{\AMC@CI@cas{\AMCIntervalFormat{#1}{#2}}}
\def\AMCIntervals#1#2#3#4{%
\FPeval\AMC@CI@a{clip(#2)}%
\let\AMC@CI@cas=\wrongchoice%
\loop%
  \FPeval\AMC@CI@b{clip(AMC@CI@a + #4)}%
  \FPiflt{#1}\AMC@CI@b\let\AMC@CI@cas=\correctchoice\fi%
  \FPiflt{#1}\AMC@CI@a\let\AMC@CI@cas=\wrongchoice\fi%
  \@expandtwoargs\AMC@intervx{\AMC@CI@a}{\AMC@CI@b}%
\FPiflt\AMC@CI@b{#3}%
  \FPset\AMC@CI@a{\AMC@CI@b}%
\repeat}
%    \end{macrocode}\end{macro}
%
% \subsection{Open questions}
%
% \begin{macro}{\AMCOpen}
%   The command |\AMCOpen|\marg{options}\marg{choices} can be
%   used as a replacement for the |choices| environment when asking
%   the student to write some answer by hand. The teacher will correct
%   and mark this answer either on the paper before scanning, or with
%   manual data capture, thanks to the scoring boxes.
%
%   As an example,
%\iffalse
%<*doc>
%\fi
\begin{verbatim}
\begin{question}{Linux}
  What is the first name of the person who started working on the Linux kernel?
  \AMCOpen{}{\wrongchoice[w]{w}\scoring{0}\correctchoice[c]{c}\scoring{2}}
\end{question}
\end{verbatim}
% \iffalse
%</doc>
%\fi
% shows:
%
% \begin{center}
% {\makeatletter\AMC@inside@boxfalse\AMC@ensemblefalse\makeatother
% \fbox{
% \begin{minipage}{.7\linewidth}
% \begin{question}{Linux}
%   What is the first name of the person who started working on the Linux kernel?
%   \AMCOpen{}{\wrongchoice[w]{w}\scoring{0}\correctchoice[c]{c}\scoring{2}}%
% \end{question}%
% \end{minipage}}}
% \end{center}
% The teacher will have to tick the `w' box for wrong answers, and the
% `c' box for correct answers.
%
% Begin with the options definitions:
%    \begin{macrocode}
\def\AMCotextGoto{}
\def\AMCotextReserved{}
\def\AMCocol@Background{lightgray}
\def\AMCocol@BoxFrameRule{white}
\def\AMCocol@FrameRule{black}
\def\AMCocol@Foreground{}
\def\AMCopen@answer{}
\def\AMCopen@question{}
\define@key{AMCOpen}{backgroundcol}{\def\AMCocol@Background{#1}}
\define@key{AMCOpen}{foregroundcol}{\def\AMCocol@Foreground{#1}}
\define@key{AMCOpen}{Treserved}{\def\AMCotextReserved{#1}}
\define@key{AMCOpen}{question}{\def\AMCopen@question{#1}}
\define@key{AMCOpen}{answer}{\def\AMCopen@answer{#1}}
\newdimen\AMCopen@Hspace\AMCopen@Hspace=.5em
\define@key{AMCOpen}{hspace}{\AMCopen@Hspace=#1}
\def\AMCopen@Width{.95\linewidth}
\define@key{AMCOpen}{width}{\def\AMCopen@Width{#1}}
\newdimen\AMCopen@LineHeight\AMCopen@LineHeight=1cm
\define@key{AMCOpen}{lineheight}{\AMCopen@LineHeight=#1}
\newcount\AMCopen@Lines\AMCopen@Lines=1
\define@key{AMCOpen}{lines}{\AMCopen@Lines=#1}
\newdimen\AMCopen@boxmargin\AMCopen@boxmargin=3pt
\define@key{AMCOpen}{boxmargin}{\AMCopen@boxmargin=#1}
\newdimen\AMCopen@boxframerule\AMCopen@boxframerule=1pt
\define@key{AMCOpen}{boxframerule}{\AMCopen@boxframerule=#1}
\define@key{AMCOpen}{boxframerulecol}{\def\AMCocol@BoxFrameRule{#1}}
\define@key{AMCOpen}{framerulecol}{\def\AMCocol@FrameRule{#1}}
\newdimen\AMCopen@framerule\AMCopen@framerule=1pt
\define@key{AMCOpen}{framerule}{\AMCopen@framerule=#1}
\define@boolkey{AMCOpen}{dots}[true]{}
\define@boolkey{AMCOpen}{scan}[true]{}
\define@boolkey{AMCOpen}{annotate}[false]{}
\define@boolkey{AMCOpen}{lineup}[false]{}
\setkeys{AMCOpen}{dots,scan,annotate,lineup}
\newcommand\AMCopenOpts[1]{\setkeys{AMCOpen}{#1}}
%    \end{macrocode}
% 
% The command |\AMCOpen| is similar to |\AMCnumericChoices|,
% calling either |\AMCopenShow| or |\AMCopenHide|.
%    \begin{macrocode}
\newcommand\AMCopenShow[4]{
  \ifAMC@ensemble\def\AMCid@name{#3}\AMCid@quest=#4\fi%
  \ifAMC@ensemble\ifAMCformulaire@dedans%
    \AMC@amclog{AUTOQCM[Q=\the\AMCid@quest]^^J}%
  \fi\fi%
  {\setkeys{AMCOpen}{#1}%
    \ifKV@AMCOpen@lineup%
      \ifAMC@ensemble\else\par\fi%
      \ifAMC@correc\smash{\AMCopen@answer}\fi\dotfill%
    \else%
      \linebreak[1]\hspace*{\fill}%
    \fi%
    {\AMCnoCompleteMulti%
      \def\AMCbeginAnswer{}\def\AMCendAnswer{}%
      \def\AMCanswer##1##2{##1\ifAMC@ensemble\else%
          \ifAMC@inside@box\else{\hspace{\AMCcodeBoxSep}\textbf{##2}}\fi\fi%
        \hspace{\AMCopen@Hspace}}%
      \fboxsep=\AMCopen@boxmargin%
      \fboxrule=\AMCopen@boxframerule%
      \fcolorbox{\AMCocol@BoxFrameRule}{\AMCocol@Background}{%
        \ifAMC@ensemble\AMCopen@question%
          \ifx\@empty\AMCopen@question\@empty\else\hspace{\AMCopen@Hspace}\fi%
        \fi%
        \begin{choicescustom}[o]%
          \ifx\AMCocol@Foreground\@empty\@empty\else%
            \def\AMC@boxcolor{\AMCocol@Foreground}%
          \fi%
          #2%
          \ifKV@AMCOpen@scan\else\AMCdontScan\fi%
          \ifKV@AMCOpen@annotate\else\AMCdontAnnotate\fi%
        \end{choicescustom}%
        \ifx\@empty\AMCotextReserved\@empty%
          \hspace{-\AMCopen@Hspace}%
        \else%
          \ifx\AMCocol@Foreground\@empty\@empty%
            \AMCotextReserved%
          \else%
            \textcolor{\AMCocol@Foreground}{\AMCotextReserved}%
          \fi%
        \fi%
      }}%
    \ifKV@AMCOpen@lineup\else%
      \par\noindent%
      \hspace*{\fill}{%
        \fboxrule=\AMCopen@framerule%
        \fcolorbox{\AMCocol@FrameRule}{white}{%
          \begin{minipage}{\AMCopen@Width}%
            \loop\vspace{\AMCopen@LineHeight}
              ~\ifAMC@correc\smash{\AMCopen@answer}\def\AMCopen@answer{}\fi%
              \ifKV@AMCOpen@dots%
                \dotfill~
              \fi
            \ifnum\AMCopen@Lines>\@ne\par\advance\AMCopen@Lines\m@ne\repeat%
          \end{minipage}}}%
      \vspace{7mm}\par%
    \fi%
  }%
  \ifAMC@ensemble\ifAMCformulaire@dedans%
  \AMC@amclog{AUTOQCM[FQ]^^J}%
  \fi\fi%
}
\newcommand\AMCopenHide[4]{
  \AMCotextGoto%
  \vspace{1.5ex}\par%
}
\def\AMCOpen{\AMCformatChoices{\AMCopenShow}{\AMCopenHide}}
%    \end{macrocode}
% \end{macro}
%
% \subsection{Page formatting}
%
% \subsubsection{Watermark}
%
% \begin{macro}{\AMCw@termark}\begin{macro}{\AMCw@terprint}
%     These commands are used to print a grey ``DRAFT'' under each
%     page, so as to prevent from printing old versions of the
%     subject.
%    \begin{macrocode}
\DeclareFontShape{OT1}{cmr}{b}{n}{<35->cmr17}{}
\def\AMC@watertext{\AMC@loc@draft}
\newcommand\AMCw@termark{%
  \setlength{\@tempdimb}{.5\paperwidth}%
  \setlength{\@tempdimc}{-.5\paperheight}%
  \put(\strip@pt\@tempdimb,\strip@pt\@tempdimc){%
    \makebox(0,0){\rotatebox{45}{\AMC@LR{%
        \textcolor[gray]{0.8}{
          \fontencoding{OT1}\fontfamily{cmr}
          \fontseries{b}\fontshape{n}
          \fontsize{90pt}{120pt}
          \selectfont
          \AMC@watertext}}}}}}
\newcommand\AMCw@terprint[1]{%
  \setbox\@tempboxa\vbox to \z@{%
    \vbox{%
      \hbox to \z@{%
        #1\hss}}\vss}
  \dp\@tempboxa\z@
  \box\@tempboxa}
%    \end{macrocode}\end{macro}\end{macro}
%
% \subsubsection{Signs for scan analysis}
%
% The following code sets up all the signs to be printed on the pages
% so as to be able to recognize the position of the boxes on the
% scans. Four circles \makeatletter\m@rqueCalage\makeatother{} are
% printed on the corners (see |\m@rqueCalage|), and binary boxes show
% the student sheet number, the page and a checking number.
%
% |\AMC@intituleHead| is the title to be printed at the beginning
% (used for corrected sheet, and empty on subject). |\AMC@note| is
% printed at the bottom of each page.
%
%    \begin{macrocode}
\def\AMCcercle#1#2{%
  {\setlength{\unitlength}{1mm}%
    \begin{picture}(#1,#1)(-#2,-#2)\thinlines\circle*{#1}\end{picture}}}
\def\m@rqueCalage{\AMCcercle{3.6}{1.8}}
\def\m@rque#1{\AMC@tracebox{1}{#1}{\m@rqueCalage}}
\def\he@dtaille#1{\vbox to 1cm{#1}}
\def\he@dbas#1{\he@dtaille{\vspace*{\fill}#1}}
\def\he@dhaut#1{\he@dtaille{#1\vspace*{\fill}}}
\def\AMC@intituleHead{\AMC@loc@corrected}
\def\AMC@note{}
\AtBeginPage{\ifAMC@pagelayout\global\advance\AMCid@check\m@ne%
  \ifnum\AMCid@check<1\global\AMCid@check=\AMCid@checkmax\fi%
  \AMC@pagepos%
  \ifAMC@watermark\ifAMC@correchead\else\AMCw@terprint{\AMCw@termark}%
  \fi\fi\fi}
\AtBeginDocument{%
  \ifAMC@pagelayout%
  \pagestyle{fancy}
  \renewcommand{\headrulewidth}{0pt}
  \ifAMC@correchead
    \fancyhf{}\fancyhead[C]{\textsc{\AMC@intituleHead}}
  \else
    \fancyhf{}
    \fancyhead[L]{\AMC@LR{\he@dbas{\leavevmode\m@rque{positionHG}}}}
    \fancyhead[R]{\AMC@LR{\he@dbas{\leavevmode\m@rque{positionHD}}}}
    \fancyfoot[L]{\AMC@LR{\leavevmode\m@rque{positionBG}}}
    \fancyfoot[R]{\AMC@LR{\leavevmode\m@rque{positionBD}}}
    \fancyhead[C]{\AMC@LR{\he@dhaut{%
        \begin{minipage}[b]{\AMC@CBtaille}\AMCboxColor{black}%
          \ifAMCids@top\vbox to \AMCids@height{\texttt{+\the\AMCid@etud/\thepage/\the\AMCid@check+}}\fi%
          \AMCbin@begin{1}\noindent%
          \AMC@binaryBoxes[\AMC@NCBetud]{\the\AMCid@etud}\\
          \AMCbin@begin{2}\noindent%
          \AMC@binaryBoxes[\AMC@NCBpage]{\thepage}\ignorespaces%
          \AMCbin@begin{3}\AMC@binaryBoxes[\AMC@NCBcheck]{\the\AMCid@check}%
        \end{minipage}%
        \ifAMCids@side\hbox to \AMCids@width{\hspace*{\fill}%
          \texttt{+\the\AMCid@etud/\thepage/\the\AMCid@check+}}\fi%
      }}}
    \fancyhfoffset[EOLR]{5mm}
  \fi
  \fancyfoot[C]{\AMC@note}
  \fi
}
%    \end{macrocode}
%
% \subsection{Defining a single exam copy content}
%
% \begin{macro}{\onecopy}
%   The command |\onecopy|\oarg{n}\marg{code} generates \meta{n}
%   copies of the subject that is described in \meta{code}. The \LaTeX
%   code \meta{code} that generates a single copy can be a little
%   long, so that the environment |examcopy| is often prefered.
%
%    \begin{macrocode}
\newcommand{\onecopy}[2]{%
  \ifx\AMCNombreCopies\undefined\AMCnum@copies=#1%
  \else\AMCnum@copies=\AMCNombreCopies\fi%
  \AMC@amclog{AUTOQCM[TOTAL=\the\AMCnum@copies]^^J}%
  \AMCid@etud=\AMCid@etudstart%
  \ifnum\AMCid@etud=0\AMCid@etud=\AMC@premierecopie\fi%
  \AMCid@etudfin=\AMCnum@copies%
  \advance\AMCid@etudfin\AMCid@etud\relax%
  \ifAMC@correchead\AMCid@etudfin=\AMC@premierecopie\fi
  \loop{%
    \AMC@zoneformulairefalse\setcounter{page}{1}%
    \AMCnumero{1}%
    \ifAMC@calibration\AMC@amclog{AUTOQCM[ETU=\the\AMCid@etud]^^J}\fi%
    #2\clearpage}\advance\AMCid@etud\@ne\ifnum\AMCid@etud<\AMCid@etudfin\repeat%
  \global\AMCid@etudstart=\AMCid@etud%
}
%    \end{macrocode}\end{macro}
%
% \begin{macro}{\AMCcleardoublepage}
%   If you want to print the subject all at one time in duplex mode,
%   it is necessary to end each subject with an even number of
%   pages. This can be achieved using |\AMCcleardoublepage| at the end
%   of the copy definition. This command is also useful inserted
%   before the separate answer sheet (if any).
%
%    \begin{macrocode}
\def\AMCcleardoublepage{\ifodd\thepage\clearpage~\fi\clearpage}
%    \end{macrocode}\end{macro}
%
% \begin{macro}{\exemplairepair}
%   To make some differences in the copies, checking if the student
%   sheet number is odd, with |\exemplairepair| construct, can be
%   useful.
%    \begin{macrocode}
\def\exemplairepair{\ifodd\AMCid@etud}
%    \end{macrocode}\end{macro}
%
% \begin{macro}{\AMClabel}\begin{macro}{\AMCref}\begin{macro}{\AMCref}
%       Commands |\AMClabel|, |\AMCref| and |\AMCpageref| replaces
%       \LaTeX's |\label|, |\ref| and |\pageref| to be able to use different
%       labels for different sheets.
%
%    \begin{macrocode}
\def\AMClabel#1{\expandafter\label{\the\AMCid@etud-#1}}
\def\AMCref#1{\expandafter\ref{\the\AMCid@etud-#1}}
\def\AMCpageref#1{\expandafter\pageref{\the\AMCid@etud-#1}}
%    \end{macrocode}\end{macro}\end{macro}\end{macro}
%
% \subsection{Package options}
%
% See section~\ref{d:options} for the options descriptions.
%
%    \begin{macrocode}
\DeclareOptionX{noshuffle}{\AMC@ordretrue}
\DeclareOptionX{answers}{\AMC@correcheadtrue\AMC@correctrue}
\DeclareOptionX{indivanswers}{\AMC@correctrue}
\DeclareOptionX{box}{\AMC@qbloctrue}
\DeclareOptionX{separateanswersheet}{\AMC@ensembletrue}
\DeclareOptionX{digits}{\AMC@inside@digittrue}
\DeclareOptionX{ordre}{\AMC@ordretrue}
\DeclareOptionX{correc}{\AMC@correcheadtrue\AMC@correctrue}
\DeclareOptionX{modele}{\AMC@correcheadtrue\AMC@correcfalse\AMC@ordretrue}
\DeclareOptionX{correcindiv}{\AMC@correctrue}
\DeclareOptionX{init}{\AMC@SR@time}
\DeclareOptionX{bloc}{\AMC@qbloctrue}
\DeclareOptionX{completemulti}{\AMCcomplete@multitrue}
\DeclareOptionX{insidebox}{\AMC@inside@boxtrue}
\DeclareOptionX{ensemble}{\AMC@ensembletrue}
\DeclareOptionX{chiffres}{\AMC@inside@digittrue}
\DeclareOptionX{outsidebox}{\AMC@outside@boxtrue}
\DeclareOptionX{calibration}{\AMC@calibrationtrue}
\DeclareOptionX{nowatermark}{\AMC@watermarkfalse}
\DeclareOptionX{catalog}{\AMC@watermarkfalse\AMC@correcheadtrue%
  \AMC@correctrue\AMC@ordretrue%
  \def\AMC@intituleHead{\AMC@loc@catalog}\AMC@affichekeystrue}
\DeclareOptionX{francais}{\AMC@loc@FR}
\DeclareOptionX{lang}{\csname AMC@loc@#1\endcsname}
\DeclareOptionX{versionA}{%
  \def\AMCid@checkmax{31}\def\AMC@NCBetud{9}\def\AMC@NCBpage{4}%
  \def\AMC@NCBcheck{5}\setlength{\AMC@CBtaille}{4cm}%
  \def\AMC@premierecopie{100}}
\DeclareOptionX{plain}{\AMC@plaintrue}
\DeclareOptionX{nopage}{\AMC@pagelayoutfalse}
\DeclareOptionX{postcorrect}{\AMC@postcorrecttrue}
\ProcessOptionsX
%    \end{macrocode}
%
% \subsection{Optional features}
%
% This package tries to see if optional packages \textsf{environ} and
% \textsf{etex} are loadable, and load them if possible.  This behaviour
% can be cancelled by using |plain| option.
%
%    \begin{macrocode}
\ifAMC@plain
\else
  \IfFileExists{environ.sty}{\RequirePackage{environ}}{}
  \ifx\eTeXversion\@undefined
  \else
    \RequirePackage{etex}
  \fi
\fi
%    \end{macrocode}
%
% \begin{environment}{examcopy}
%   Then, if \textsf{environ} package is loaded and defines command
%   |\NewEnviron|, environment |examcopy| is defined.
%
%   Environment |{examcopy}|\oarg{n} does the same as command
%   |onecopy|: it encloses \LaTeX{} code which makes \emph{one} exam
%   copy. Optional argument \meta{n} gives the number of desired
%   copies -- this can also be modified redefinig |\AMCNombreCopies|.
%
%    \begin{macrocode}
\@ifpackageloaded{environ}{%
  \ifx\NewEnviron\undefined\PackageWarning{automultiplechoice}%
  {Package environ loaded but too old version:
    environnement examcopy/copieexamen will NOT be defined.}%
  \else\NewEnviron{examcopy}[1][5]{\onecopy{#1}{\BODY}}\fi}%
{\PackageWarning{automultiplechoice}%
  {Package environ not loaded: environnement
    examcopy/copieexamen will NOT be defined.}}
%    \end{macrocode}\end{environment}
%
% \subsection{External control}
%
% \begin{macro}{\SujetExterne}\begin{macro}{\CalibrationExterne}
% \begin{macro}{\CorrigeExterne}\begin{macro}{\CorrigeIndivExterne}
% \begin{macro}{\NoWatermarkExterne}
%   Some of the package options can be controlled defining
%   |\|$xxx$|Externe| commands. For example, the following command will
%   format the subject document, whatever options are used in the
%   \LaTeX{} file:
%\iffalse
%<*doc>
%\fi
\begin{verbatim}
pdflatex '\nonstopmode\def\SujetExterne{1}\def\NoWatermarkExterne{1}\input{mcq.tex}'
\end{verbatim}
%\iffalse
%</doc>
%\fi
%
%    \begin{macrocode}
\ifx\SujetExterne\undefined\else
\message{***SUJET***^^J}
\AMC@calibrationtrue\AMC@correcfalse\AMC@correcheadfalse\AMC@watermarkfalse
\fi
\ifx\CalibrationExterne\undefined\else
\message{***CALIBRATION***^^J}
\AMC@calibrationtrue\AMC@correcfalse\AMC@correcheadfalse\AMC@watermarkfalse
\fi
\ifx\CorrigeExterne\undefined\else
\message{***CORRIGE***^^J}
\AMC@calibrationfalse\AMC@correcheadtrue\AMC@correctrue\AMC@watermarkfalse
\fi
\ifx\CorrigeIndivExterne\undefined\else
\message{***CORRIGE***^^J}
\AMC@calibrationfalse\AMC@correcheadfalse\AMC@correctrue\AMC@watermarkfalse
\fi
\ifx\NoWatermarkExterne\undefined\else
\AMC@watermarkfalse
\fi
%    \end{macrocode}
% \end{macro}\end{macro}\end{macro}\end{macro}\end{macro}
%
% \subsection{Page layout}
%
% The following code sets the correct page layout to have room for
% signs for scan analysis, and prepares watermark printing:
%    \begin{macrocode}
\@ifpackageloaded{geometry}{}{\usepackage{geometry}}
\ifAMC@pagelayout
  \ifAMC@correchead
    \geometry{hmargin=3cm,vmargin={1cm,1cm},includeheadfoot,headheight=1cm,footskip=1cm}
  \else
    \geometry{hmargin=3cm,headheight=2cm,headsep=.3cm,footskip=1cm,top=3.5cm,bottom=2.5cm}
  \fi
  \ifAMC@watermark
    \ifAMC@correchead\else
      \def\AMC@note{\begin{minipage}{0.65\linewidth}
          \AMC@LR{\textcolor{blue}{\AMC@loc@message}}
        \end{minipage}
      }
    \fi
  \fi
\fi
%    \end{macrocode}
%
% \subsection{Initialisation}
%
% Initialisation of the check counter:
%    \begin{macrocode}
\AMCid@check=\AMCid@checkmax\advance\AMCid@check\@ne
%    \end{macrocode}
%
% Telling outside if separate answer sheet, and boxes labelling, are requested:
%    \begin{macrocode}
\ifAMC@ensemble\AMC@amclog{AUTOQCM[VAR:ensemble=1]^^J}\fi
\ifAMC@inside@box\AMC@amclog{AUTOQCM[VAR:insidebox=1]^^J}\fi
\ifAMC@outside@box\AMC@amclog{AUTOQCM[VAR:outsidebox=1]^^J}\fi
\ifAMC@postcorrect\AMC@amclog{AUTOQCM[VAR:postcorrect=1]^^J}\fi
%    \end{macrocode}
%
% Preparing writing to |.xy| file :
%    \begin{macrocode}
\ifAMC@calibration
\newwrite\AMC@XYFILE%
\immediate\openout\AMC@XYFILE\jobname.xy%
\immediate\write\AMC@XYFILE{\string\version{\AMC@VERSION}}
\fi
%    \end{macrocode}
%
% \subsection{French command names}
%
% For backward compatibility, a lot of commands have their french
% counterpart:
%    \begin{macrocode}
\let\reponses=\choices\let\endreponses=\endchoices
\let\reponseshoriz=\choiceshoriz\let\endreponseshoriz=\endchoiceshoriz
\let\reponsesperso=\choicescustom\let\endreponsesperso=\endchoicescustom
\let\bonne=\correctchoice
\let\mauvaise=\wrongchoice
\let\bareme=\scoring
\let\baremeDefautM=\scoringDefaultM
\let\baremeDefautS=\scoringDefaultS
\def\exemplaire{\AMC@loc@FR\onecopy}
\@ifpackageloaded{environ}{%
  \let\copieexamen=\examcopy\let\endcopieexamen=\endexamcopy}{}
\let\melangegroupe=\shufflegroup
\let\restituegroupe=\insertgroup
\let\alafin=\lastchoices
\let\formulaire=\AMCform
\let\AMCdebutFormulaire=\AMCformBegin
\let\champnom=\namefield
\let\choixIntervalles=\AMCIntervals
%    \end{macrocode}
%
% \section{Outputs}
%
% In the |.xy| file, |1/|\meta{n} means
% student sheet number~1 (there is only one ``student sheet'' for this
% document as we did not use |\onecopy|) and page number~\meta{n} inside this
% student sheet. Then, each instance of the |\tracepos| command shows
% $x$ and $y$ positions as arguments \#2 and \#3 (unit is |sp|, such
% that $65536\times 72.27\,|sp|$ is one inch). One has to take min and
% max of the $x$-values to determine the left and right position of
% the box, and min and max values of $y$-values to determine top and bottom
% position of the box.
%
% \subsection{\texttt{namefield} command}
% \label{a:name}
% Lines in the |.xy| file from a |\namefield| command:
% \verbatiminput{automultiplechoice.xy2} 
%
% \subsection{\texttt{AMCboxedchar} command}
% \label{a:boxed}
% Lines in the |.xy| file from a |\AMCboxedchar| command:
% \verbatiminput{automultiplechoice.xy1} 
%
% \subsection{\texttt{AMCcode} command}
% \label{a:code}
% Lines in the |.xy| file from a |\AMCcode| command. Here,
% |code.|\meta{n}|:|\meta{q}|,|\meta{v} relates to digit number
% \meta{n} from the right (\meta{n}=1 for units, \meta{n}=2 for tens,
% \meta{n}=3 for hundreds and so on), question number \meta{q}
% (|\AMCcode| uses a fake question; this number can be ignored), and
% value \meta{v}-1 (box number \meta{v} for the digit).
% \verbatiminput{automultiplechoice.xy3}
%
% \clearpage
% \tableofcontents
%
% \clearpage
% \PrintIndex
%
% \end{document}
