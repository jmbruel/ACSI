\documentclass{article}

\usepackage[bloc,completemulti]{automultiplechoice}

%% XeLaTeX with arabic texts, using Rasheeq font from arabeyes:
\usepackage{arabxetex}
\newfontfamily{\arabicfont}[Script=Arabic,Scale=1]{Rasheeq}
%%

\begin{document}

%%%%%%%%%%%%%%%%%%%%%%%%%%%%%%%%%%%%%%%%%%%%%%%%%%%%%%
%% some commands to turn texts into arabic :

\Arabic

\def\AMCchoiceLabel#1{\textLR{\Alph{#1}}}
\def\AMCbeginQuestion#1#2{\par\noindent{‫السؤال #1 #2 :}}
\def\AMCformQuestion#1{\vspace{\AMCformVSpace}\par{\large ‫السؤال #1 :}}

\AMCtext{none}{لا شيء من الإجابات أعلاه صحيحة}

%%%%%%%%%%%%%%%%%%%%%%%%%%%%%%%%%%%%%%%%%%%%%%%%%%%%%%

\element{qqs}{
\begin{question}{good choice}
  كم نقطة تريد لهذا السؤال؟
  \begin{choices}
    \correctchoice{الاعلى: 10}\scoring{10}
    \wrongchoice{فقط 5}\scoring{5}
    \wrongchoice{نقطتين تكفي}\scoring{2}
    \wrongchoice{لا شئ، شكرا}\scoring{0}
  \end{choices}
\end{question}
}

\element{qqs}{
\begin{questionmult}{added}
  وضع درجة مختلفة لكل اجابة:
  \begin{choices}
    \correctchoice{2 نقطة}\scoring{b=2}
    \wrongchoice{نقطة سالبة واحدة!}\scoring{b=0,m=-1}
    \correctchoice{3 نقاط}\scoring{b=3}
    \correctchoice{1 نقطة}
    \correctchoice{نصف نقطة}\scoring{b=0.5}
  \end{choices}
\end{questionmult}
}

\element{qqs}{
\begin{questionmult}{3 or zero}\scoring{mz=3}
  الاجابة الصحيحة الكاملة فقط تحصل على 3 نقاط، اما غير ذلك فتحصل على صفر.
  \begin{choices}
    \wrongchoice{خطأ}
    \wrongchoice{خطأ}
    \correctchoice{صح}
    \correctchoice{صح}
  \end{choices}
\end{questionmult}
}

\element{qqs}{
\begin{questionmult}{all for 2}\scoring{haut=2}
  الاجابة الصحيحة الكاملة تحصل على نقطتين، وكل اجابة خاطئة تخصم نقطة...
  \begin{choices}
    \correctchoice{صح}
    \correctchoice{هذه الاجابة صحيحة}
    \correctchoice{نعم!}
    \wrongchoice{خطأ!}
    \wrongchoice{لا تختر هذه!}
  \end{choices}
\end{questionmult}
}

\element{qqs}{
\begin{question}{attention}\scoring{b=2}
  الاجابة الصحيحة تحصل على نقطتين، وكذلك الاجابة السيئة جدا تخصم نقطتين.
  \begin{choices}
    \correctchoice{جيد!}
    \wrongchoice{غير صحيحة}
    \wrongchoice{غير صحيحة}
    \wrongchoice{غير صحيحة}
    \wrongchoice{اجابة سيئة جدا!}\scoring{-2}
  \end{choices}
\end{question}
}

\element{qqs}{
\begin{questionmult}{as you like}
  اختر النقاط التي تريد:
  \begin{choices}
    \correctchoice{ستحصل على نقطتين}\scoring{b=2}
    \wrongchoice{ستحصل على 3 نقاط}\scoring{b=0,m=3}
    \correctchoice{ستحصل على نقطة اذا علمتها، وستخسر نقطة اذا تركتها}\scoring{m=-1}
  \end{choices}
\end{questionmult}
}

%%%%%%%%%%%%%%%%%%%%%%%%%%%%%%%%%%%%%%%%%%%%%%%%%%%%%%%%%%%%%%%%%%%%%%

\onecopy{20}{

\noindent{\bf اختبار  \hfill توزيع النقاط بشكل تفصيلي}

\vspace*{.5cm}
\begin{minipage}{.4\linewidth}
\centering\large اختبار\\ يناير 2011\end{minipage}
\namefield{\fbox{\begin{minipage}{.5\linewidth}
الاسم:

\vspace*{.5cm}\dotfill
\vspace*{1mm}
\end{minipage}}}
\vspace{1cm}

%%%%%%%%%%%%%%%%%%%%%%%%%%%%%%%%%%%%%%%%%%%%%%%%%%%%%%%%%%%%%%%%%%%%%%

\shufflegroup{qqs}

\insertgroup{qqs}

%%%%%%%%%%%%%%%%%%%%%%%%%%%%%%%%%%%%%%%%%%%%%%%%%%%%%%%%%%%%%%%%%%%%%%

\clearpage

}

\end{document}
